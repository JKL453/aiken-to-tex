\documentclass[11pt]{exam}

\usepackage{amsmath}
\usepackage{graphicx}
\usepackage{geometry}
\usepackage{etoolbox}
\BeforeBeginEnvironment{choices}{\par\nopagebreak\minipage{\linewidth}}
\AfterEndEnvironment{choices}{\endminipage}
\geometry{
a4paper,
total={185mm,257mm},
left=10mm,
top=25mm,
bottom=10mm
}

\begin{document}
\setlength{\voffset}{-0.5in}
\setlength{\headsep}{5pt}

\fbox{\fbox{\parbox{8cm}{\centering
\vspace{2mm}
Testat - Versuch A - Waermelehre - 1
\vspace{2mm}
}}}
\hspace{2mm}
\makebox[0.25\textwidth]{Name:\enspace\hrulefill} \hspace{5mm}
\makebox[0.2\textwidth]{Datum:\enspace\hrulefill}
\vspace{4mm}

\begin{questions}

\question Wasser hat eine spezifische Wärmekapazität von (sehr grob angenähert) 4 kJ kg\(^{-1}\) K\(^{-1}\) und eine spezifische Verdampfungswärme von etwa 2 MJ kg\(^{-1}\). Einem Kilogramm Wasser wird bei einer Temperatur von 99\(^\circ\)C eine Wärmemenge von 2 MJ zugeführt. Als Ergebnis erhält man

\begin{choices}
	\choice etwa 1 kg Wasserdampf. (correct)
	\choice etwa 1 kg heißes Wasser.
	\choice etwa 0,5 kg heißes Wasser und 0,5 kg Wasserdampf.
	\choice etwa 1 kg heißes Wasser und 1 kg Wasserdampf.
	\choice etwa 2 kg heißes Wasser.
\end{choices}

\vspace{3mm}\question Wenn warme und kalte Flüssigkeiten in gleichen Mengen gemischt werden, dann hat das Gemisch

\begin{choices}
	\choice eine Wärmeenergie, die gleich der Summe der Wärmeenergien der beiden Flüssigkeiten ist. (correct)
	\choice die Wärmeenergie der kalten Flüssigkeit.
	\choice eine Wärmeenergie kleiner als die Wärmeenergie der warmen Flüssigkeit.
	\choice die Wärmeenergie der warmen Flüssigkeit.
	\choice eine Wärmeenergie kleiner als die Wärmeenergie der kalten Flüssigkeit.
\end{choices}

\vspace{3mm}\question Wenn eine warme und eine kalte Flüssigkeitsmenge gleicher Art gemischt werden, dann hat das Gemisch

\begin{choices}
	\choice eine spezifische Wärmekapazität etwa gleich der Summe der spezifischen Wärmekapazitäten der beiden Flüssigkeiten.
	\choice die doppelte spezifische Wärmekapazität der kalten Flüssigkeit.
	\choice die halbe spezifische Wärmekapazität der kalten Flüssigkeit.
	\choice die halbe spezifische Wärmekapazität der warmen Flüssigkeit.
	\choice eine spezifische Wärmekapazität, die ähnlich groß wie die spezifischen Wärmekapazität der kalten oder der warmen Flüssigkeit ist. (correct)
\end{choices}

\vspace{3mm}\question Wasser hat eine spezifische Wärmekapazität von (sehr grob angenähert) 4 kJ kg\(^{-1}\) K\(^{-1}\) und eine spezifische Verdampfungswärme von etwa 2 MJ kg\(^{-1}\).Einem Kilogramm Wasser wird bei einer Temperatur von 99\(^\circ\)C eine Wärmemenge von 2 MJ zugeführt. Als Ergebnis erhält man

\begin{choices}
	\choice etwa 1 kg heißes Wasser und 1 kg Wasserdampf.
	\choice etwa 0,5 kg heißes Wasser und 0,5 kg Wasserdampf.
	\choice etwa 2 kg heißes Wasser.
	\choice etwa 1 kg heißes Wasser.
	\choice etwa 1 kg Wasserdampf. (correct)
\end{choices}

\vspace{3mm}\question Welcher Zusammenhang besteht zwischen Wärme, Wärmekapazität und Temperatur? (\(m\): Masse)

\begin{choices}
	\choice \(\Delta Q = m \, (\Delta T)^2\)
	\choice \(\Delta Q = m \Delta T\)
	\choice \(\Delta Q = m c \Delta T\) (correct)
	\choice \(Q = m c^2\)
	\choice \(Q = m c\)
\end{choices}

\vspace{3mm}\end{questions}

\end{document}
