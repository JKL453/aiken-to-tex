\documentclass[11pt]{exam}

\usepackage{amsmath}
\usepackage{graphicx}
\usepackage{geometry}
\usepackage{etoolbox}
\BeforeBeginEnvironment{choices}{\par\nopagebreak\minipage{\linewidth}}
\AfterEndEnvironment{choices}{\endminipage}
\geometry{
a4paper,
total={185mm,257mm},
left=10mm,
top=25mm,
bottom=10mm
}

\begin{document}
\setlength{\voffset}{-0.5in}
\setlength{\headsep}{5pt}

\fbox{\fbox{\parbox{8cm}{\centering
\vspace{2mm}
Testat - Versuch G - Geometrische Optik - 2
\vspace{2mm}
}}}
\hspace{2mm}
\makebox[0.25\textwidth]{Name:\enspace\hrulefill} \hspace{5mm}
\makebox[0.2\textwidth]{Datum:\enspace\hrulefill}
\vspace{4mm}

\begin{questions}

\question Eine Sammellinse hat eine Brechkraft von 3 dpt. In welcher Entfernung entsteht das Bild, wenn der Gegenstand 100 cm entfernt ist?

\begin{choices}
	\choice 100 cm
	\choice 50 cm (correct)
	\choice 33,3 cm
	\choice 150cm
	\choice 75 cm
\end{choices}

\vspace{3mm}\question Ein in Luft normalsichtiger kann unter Wasser ohne Hilfsmittel (z.B. Taucherbrille) nicht scharf sehen. Die Ursache hierfür ist:

\begin{choices}
	\choice Trübung der Cornea durch eindringendes Wasser.
	\choice Verstärkung der Lichtstreuung an der Cornea-Vorderseite.
	\choice Zunahme der Cornea-Krümmung durch den Wasserdruck.
	\choice Abnahme der Cornea-Krümmung durch den Wasser-Druck.
	\choice Verminderung der Lichtbrechung an der Cornea-Vorderfläche. (correct)
\end{choices}

\vspace{3mm}\question Wie sind die Lichtgeschwindigkeit c, die Vakuum-Lichtgeschwindigkeit c0 und der Brechungsindex n miteinander verknüpft?

\begin{choices}
	\choice n·c0 =c
	\choice 1 /C0−1/C =n
	\choice c0 − c = n
	\choice 1 /C0−1/C =1/n
	\choice n·c=c0 (correct)
\end{choices}

\vspace{3mm}\question Wie lautet die Einheit der Brechkraft?

\begin{choices}
	\choice m
	\choice s
	\choice 1/m (correct)
	\choice einheitenlos
	\choice N/m
\end{choices}

\vspace{3mm}\question Die Brennweite einer Linse sei f = 20 cm, Gegenstandsweite sei g = 40 cm, der Schirm befindet sich in der Entfernung b = 20 cm. Wie groß muss die Brechkraft einer Korrekturlinse sein, um eine scharfe Abbildung auf dem Schirm zu erhalten? (Der Abstand zwischen den Linsen ist zu vernachlässigen)

\begin{choices}
	\choice +10/2m^−1
	\choice +5dpt
	\choice −5dpt
	\choice −2,5dpt
	\choice +2,5 dpt (correct)
\end{choices}

\vspace{3mm}\end{questions}

\end{document}
