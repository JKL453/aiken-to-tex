\documentclass[11pt]{exam}

\usepackage{amsmath}
\usepackage{graphicx}
\usepackage{geometry}
\usepackage{etoolbox}
\BeforeBeginEnvironment{choices}{\par\nopagebreak\minipage{\linewidth}}
\AfterEndEnvironment{choices}{\endminipage}
\geometry{
a4paper,
total={185mm,257mm},
left=10mm,
top=25mm,
bottom=10mm
}

\begin{document}
\setlength{\voffset}{-0.5in}
\setlength{\headsep}{5pt}

\fbox{\fbox{\parbox{8cm}{\centering
\vspace{2mm}
Testat - Versuch G - Geometrische Optik - 1
\vspace{2mm}
}}}
\hspace{2mm}
\makebox[0.25\textwidth]{Name:\enspace\hrulefill} \hspace{5mm}
\makebox[0.2\textwidth]{Datum:\enspace\hrulefill}
\vspace{4mm}

\begin{questions}

\question Eine Sammellinse hat eine Brechkraft von 3 dpt. In welcher Entfernung entsteht das Bild, wenn der Gegenstand 100 cm entfernt ist?

\begin{choices}
	\choice 100 cm
	\choice 50 cm (correct)
	\choice 33,3 cm
	\choice 150cm
	\choice 75 cm
\end{choices}

\vspace{3mm}\question Ein in Luft normalsichtiger kann unter Wasser ohne Hilfsmittel (z.B. Taucherbrille) nicht scharf sehen. Die Ursache hierfür ist:

\begin{choices}
	\choice Verstärkung der Lichtstreuung an der Cornea-Vorderseite.
	\choice Trübung der Cornea durch eindringendes Wasser.
	\choice Abnahme der Cornea-Krümmung durch den Wasser-Druck.
	\choice Zunahme der Cornea-Krümmung durch den Wasserdruck.
	\choice Verminderung der Lichtbrechung an der Cornea-Vorderfläche. (correct)
\end{choices}

\vspace{3mm}\question Wann wird ein paralleles Strahlenbündel gebrochen?

\begin{choices}
	\choice Wenn es in Luft unter senkrechtem Einfall auf die plane Seite einer Plankonkavlinse trifft.
	\choice Wenn es unter senkrechtem Einfall auf eine ebene Grenzfläche zweier transparenter Medien trifft, in denen Licht eine unterschiedliche Ausbreitungsgeschwindigkeit hat.
	\choice Wenn es unter senkrechtem Einfall auf eine gekrümmte Grenzfläche zweier transparenter Medien mit gleichem Brechungsindex trifft.
	\choice Wenn es unter senkrechtem Einfall auf eine gekrümmte Grenzfläche zweier transparenter Medien mitunterschiedlichem Brechungsindex trifft. (correct)
	\choice Wenn es unter nicht senkrechtem Einfall auf die Grenzfläche eines transparenten und eines nicht-transparenten Mediums trifft.
\end{choices}

\vspace{3mm}\question Welche Grenzfläche im Auge trägt am meisten zur Gesamtbrechkraft des optischen Systems bei?

\begin{choices}
	\choice Kammerwasser-Regenbogenhaut
	\choice Hornhaut-Kammerwasser
	\choice Linse-Glaskörper
	\choice Luft-Hornhaut (correct)
	\choice Kammerwasser-Linse
\end{choices}

\vspace{3mm}\question Eine Sammellinse von 20 cm Brennweite soll mit einer Linse kombiniert werden, um eine Gesamtbrennweite von 40 cm zu erhalten. (Beide Linsen seien ’dünn’ und einander so nah, dass der Abstand vernachlässigbar klein ist.)Welche Linse ist erforderlich?

\begin{choices}
	\choice Sammellinse mit 60 cm Brennweite
	\choice Sammellinse mit 20 cm Brennweite
	\choice Zerstreuungslinse mit -20 cm Brennweite
	\choice  Sammellinse mit 40 cm Brennweite
	\choice Zerstreuungslinse mit -40 cm Brennweite (correct)
\end{choices}

\vspace{3mm}\end{questions}

\end{document}
