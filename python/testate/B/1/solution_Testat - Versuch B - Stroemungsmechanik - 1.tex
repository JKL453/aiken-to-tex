\documentclass[11pt]{exam}

\usepackage{amsmath}
\usepackage{graphicx}
\usepackage{geometry}
\usepackage{etoolbox}
\BeforeBeginEnvironment{choices}{\par\nopagebreak\minipage{\linewidth}}
\AfterEndEnvironment{choices}{\endminipage}
\geometry{
a4paper,
total={185mm,257mm},
left=10mm,
top=25mm,
bottom=10mm
}

\begin{document}
\setlength{\voffset}{-0.5in}
\setlength{\headsep}{5pt}

\fbox{\fbox{\parbox{8cm}{\centering
\vspace{2mm}
Testat - Versuch B - Stroemungsmechanik - 1
\vspace{2mm}
}}}
\hspace{2mm}
\makebox[0.25\textwidth]{Name:\enspace\hrulefill} \hspace{5mm}
\makebox[0.2\textwidth]{Datum:\enspace\hrulefill}
\vspace{4mm}

\begin{questions}

\question Das Gesetz von Hagen und Poiseuille

\begin{choices}
	\choice sagt eine konstante Volumenstromstärke voraus, wenn sowohl Länge als auch Querschnittsfläche eines Rohrs sich halbieren.
	\choice beschreibt nur turbulente Strömungen.
	\choice gilt nur für Flüssigkeiten ohne Viskosität.
	\choice erlaubt die Berechnung des Leitwertes eines Rohres. (correct)
	\choice besagt, dass die Volumenstromstärke quadratisch mit der Druckdifferenz wächst.
\end{choices}

\vspace{3mm}\question Wie groß ist die Fläche eines Stempels, wenn eine Kraft von 20 N einen Druck von 100 Pa ausübt?

\begin{choices}
	\choice 200 qm
	\choice 2000 qm
	\choice 0,5 qm
	\choice 5 qm
	\choice 0,2 qm (correct)
\end{choices}

\vspace{3mm}\question Wenn die Querschnittsfläche eines Rohres viermal kleiner wird, dann

\begin{choices}
	\choice wird der Radius halbiert. (correct)
	\choice wird auch die Länge geviertelt.
	\choice wird der Durchmesser auch geviertelt.
	\choice wird der Radius verdoppelt.
	\choice bleibt der Durchmesser unverändert.
\end{choices}

\vspace{3mm}\question Die Volumenstromstärke

\begin{choices}
	\choice gibt die Geschwindigkeit der Füssigkeitsmoleküle an.
	\choice ist gleich dem Produkt aus Rohrquerschnitt und Rohrlänge.
	\choice ist gleich der Druckdifferenz zwischen Rohranfang und Rohrende.
	\choice gibt an, welches Volumen pro Zeiteinheit durch einen Rohrquerschnitt fließt. (correct)
	\choice ist gleich der Zeit, in der ein Flüssigkeitsmolekül durch das Rohr fließt.
\end{choices}

\vspace{3mm}\question Im menschlichen Gefäßsystem strömt Blut

\begin{choices}
	\choice wie eine Flüssigkeit mit einer Viskosität von Null.
	\choice immer turbulent.
	\choice manchmal auch turbulent. (correct)
	\choice immer laminar.
	\choice wie eine Flüssigkeit ohne innere Reibung.
\end{choices}

\vspace{3mm}\end{questions}

\end{document}
