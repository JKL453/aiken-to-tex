\documentclass[11pt]{exam}

\usepackage{amsmath}
\usepackage{graphicx}
\usepackage{geometry}
\usepackage{etoolbox}
\BeforeBeginEnvironment{choices}{\par\nopagebreak\minipage{\linewidth}}
\AfterEndEnvironment{choices}{\endminipage}
\geometry{
a4paper,
total={185mm,257mm},
left=10mm,
top=25mm,
bottom=10mm
}

\begin{document}
\setlength{\voffset}{-0.5in}
\setlength{\headsep}{5pt}

\fbox{\fbox{\parbox{8cm}{\centering
\vspace{2mm}
Testat - Versuch C - Stationaerer Strom - 1
\vspace{2mm}
}}}
\hspace{2mm}
\makebox[0.25\textwidth]{Name:\enspace\hrulefill} \hspace{5mm}
\makebox[0.2\textwidth]{Datum:\enspace\hrulefill}
\vspace{4mm}

\begin{questions}

\question Wie groß ist der Gesamtwiderstand einer Parallelschaltung von drei Widerständen mit \(R_1=\mathrm{\frac{1}{3}\,\Omega}\), \(R_2=\mathrm{\frac{1}{4}\,\Omega}\) und \(R_3=\mathrm{\frac{1}{5}\,\Omega}\)?

\begin{choices}
	\choice \(\mathrm{12\,\Omega}\)
	\choice \(\mathrm{\frac{1}{12}\,\Omega}\) (correct)
	\choice \(\mathrm{\frac{47}{60}\,\Omega}\)
	\choice \(\mathrm{\frac{60}{47}\,\Omega}\)
	\choice \(\mathrm{\frac{1}{24}\,\Omega}\)
\end{choices}

\vspace{3mm}\question Ein unbekannter Widerstand \(\mathrm{R}\) wird mit einem \(\mathrm{4\,\Omega}\)-Widerstand in Reihe an eine Spannungsquelle mit \(\mathrm{U=12\,V}\) angeschlossen. An dem unbekannten Widerstand \(\mathrm{R}\) wird dabei eine Spannung von \(\mathrm{4\,V}\) gemessen. Wie groß ist \(\mathrm{R}\)?

\begin{choices}
	\choice \(\mathrm{2\,\Omega}\) (correct)
	\choice \(\mathrm{1\,\Omega}\)
	\choice \(\mathrm{5\,\Omega}\)
	\choice \(\mathrm{3\,\Omega}\)
	\choice \(\mathrm{4\,\Omega}\)
\end{choices}

\vspace{3mm}\question Aus einem Metall mit dem spezifischen Widerstand \(\rho\) wird ein runder Draht der Länge \(\mathrm{l}\) mit der Querschnittsfläche \(\mathrm{A}\) gefertigt. Sein Widerstand \(\mathrm{R}\) ergibt sich zu \(\mathrm{R=\rho \cdot \frac{l}{A}}\). Welche Einheit trägt der spezifische Widerstand \(\rho\)?

\begin{choices}
	\choice Ohm pro Quadratmeter
	\choice Ohm pro Meter
	\choice Ohm
	\choice Ohm mal Quadratmeter
	\choice Ohm mal Meter (correct)
\end{choices}

\vspace{3mm}\question Wie groß ist der Gesamtwiderstand einer Reihenschaltung von drei Widerständen mit \(R_1=\mathrm{\frac{1}{2}\,\Omega}\), \(R_2=\mathrm{\frac{1}{3}\,\Omega}\) und \(R_3=\mathrm{\frac{1}{4}\,\Omega}\)?

\begin{choices}
	\choice \(\mathrm{9\,\Omega}\)
	\choice \(\mathrm{\frac{1}{9}\,\Omega}\)
	\choice \(\mathrm{\frac{1}{24}\,\Omega}\)
	\choice \(\mathrm{\frac{13}{12}\,\Omega}\) (correct)
	\choice \(\mathrm{\frac{12}{13}\,\Omega}\)
\end{choices}

\vspace{3mm}\question Welche der folgenden Aussagen ist falsch?

\begin{choices}
	\choice Wenn zwei unterschiedliche Widerstände parallel geschaltet werden, dann fließt durch den größeren Widerstand der kleinere Strom.
	\choice Wenn zwei unterschiedliche Widerstände in Reihe geschaltet werden, dann liegt am größeren Widerstand die größere Spannung an.
	\choice Wenn zwei unterschiedliche Widerstände in Reihe geschaltet werden, dann fließt durch beide die gleiche Stromstärke.
	\choice Wenn zwei unterschiedliche Widerstände parallel geschaltet werden, liegt an beiden die gleiche Spannung an.
	\choice Der Gesamtwiderstand einer Reihenschaltung von zwei identischen Widerständen ist doppelt so groß wie der Gesamtwiderstand einer Parallelschaltung aus denselben Widerständen. (correct)
\end{choices}

\vspace{3mm}\end{questions}

\end{document}
