\documentclass[11pt]{exam}

\usepackage{geometry}
\geometry{
a4paper,
total={185mm,257mm},
left=10mm,
top=25mm,
bottom=10mm
}

\begin{document}
\setlength{\voffset}{-0.5in}
\setlength{\headsep}{5pt}

\fbox{\fbox{\parbox{8cm}{\centering
\vspace{2mm}
Testat - Versuch E - Diffusion und Osmose - 1
\vspace{2mm}
}}}
\hspace{2mm}
\makebox[0.25\textwidth]{Name:\enspace\hrulefill} \hspace{5mm}
\makebox[0.2\textwidth]{Datum:\enspace\hrulefill}
\vspace{4mm}

\begin{questions}

\question In einer Doppelkammer, auf deren einer Seite eine Konzentration von 25 \(\frac{mg}{l} \) und auf der anderen von 5 \(  \frac{mg}{l} \) herrscht  und die durch eine Membran der Dicke 100 \(\mu \)m geteilt ist, herrscht ein Diffusionsfluss von 8  \( \frac{mg}{min}\). Wie groß ist der Diffusionsfluss, wenn die Konzentration auf der einen Seite von 5 \(  \frac{mg}{l} \)   auf 20 \(  \frac{mg}{l} \)  erhöht wird?

\begin{choices}
	\choice 16 \( \frac{mg}{min} \)
	\choice 32 \( \frac{mg}{min} \)
	\choice 2 \( \frac{mg}{min} \)
	\choice 4 \( \frac{mg}{min} \)
	\choice 8 \( \frac{mg}{min} \)
\end{choices}

\vspace{3mm}\question Die Leitfähigkeit von 15 ml Salzlösung wird zu 6 \( \frac{\mu S}{cm} \) bestimmt. Diese Salzlösung wird nun mit destilliertem Wasser auf 60 ml aufgefüllt. Welche Leitfähigkeit wird nun gemessen?

\begin{choices}
	\choice \( \frac{6}{4} \)  \( \frac{\mu S}{cm} \)
	\choice \( \frac{4}{6} \)  \( \frac{\mu S}{cm} \)
	\choice 4,25  \( \frac{\mu S}{cm} \)
	\choice \( \frac{3}{6} \)  \( \frac{\mu S}{cm} \)
	\choice 6,75  \( \frac{\mu S}{cm} \)
\end{choices}

\vspace{3mm}\question Wie ändert sich der Diffusionsfluss \( J \) durch eine kreisförmige Membran, wenn deren Durchmesser sich verdoppelt?

\begin{choices}
	\choice \( J \) halbiert sich.
	\choice \( J \) vervierfacht sich.
	\choice \( J \) quadriert sich.
	\choice \( J \) verdoppelt sich.
	\choice \( J \) ist unabhängig vom Durchmesser der Membran.
\end{choices}

\vspace{3mm}\question Welche Aussage zur Osmose trifft zu?

\begin{choices}
	\choice ... steht mit dem Phänomen der Diffusion in keinem Zusammenhang
	\choice ... hat in der Physiologie praktisch keine Bedeutung
	\choice ... lässt sich nicht direkt beobachten
	\choice ... ist ein reversibler Prozess
	\choice ... ist ein irreversibler Prozess
\end{choices}

\vspace{3mm}\question Von welchen Parametern hängt der Diffusionskoeffizient eines Stoffes (Flüssigkeit oder Gas) ab?A) molare Masse des StoffesB) Konzentration des StoffesC) TemperaturD) Art des LösungsmittelsE) Ladungszahl des Stoffes

\begin{choices}
	\choice A, B und D sind richtig
	\choice A ist falsch
	\choice C und D sind richtig
	\choice B und E sind falsch
	\choice A, B, C und E sind richtig
\end{choices}

\vspace{3mm}\end{questions}

\end{document}
