\documentclass[11pt]{exam}

\usepackage{geometry}
\geometry{
a4paper,
total={185mm,257mm},
left=10mm,
top=25mm,
bottom=10mm
}

\begin{document}
\setlength{\voffset}{-0.5in}
\setlength{\headsep}{5pt}

\fbox{\fbox{\parbox{8cm}{\centering
\vspace{2mm}
Testat - Versuch E - Diffusion und Osmose - 3
\vspace{2mm}
}}}
\hspace{2mm}
\makebox[0.25\textwidth]{Name:\enspace\hrulefill} \hspace{5mm}
\makebox[0.2\textwidth]{Datum:\enspace\hrulefill}
\vspace{4mm}

\begin{questions}

\question In einer Doppelkammer, auf deren einer Seite eine Konzentration von 25 \(\frac{mg}{l} \) und auf der anderen von 10 \(  \frac{mg}{l} \) herrscht  und die durch eine Membran der Dicke 100 \(\mu \)m geteilt ist, herrscht ein Diffusionsfluss von 4  \( \frac{mg}{min}\). Wie groß ist der Diffusionsfluss, wenn die Konzentration auf der einen Seite von 10 \(  \frac{mg}{l} \)   auf 20 \(  \frac{mg}{l} \)  erhöht wird?

\begin{choices}
	\choice \( \frac{4}{5} \) \( \frac{mg}{min} \)
	\choice \( \frac{5}{4} \) \( \frac{mg}{min} \)
	\choice \( \frac{2}{3} \) \( \frac{mg}{min} \)
	\choice \( \frac{5}{4} \) \( \frac{mg}{min} \) (correct)
	\choice \( \frac{3}{4} \) \( \frac{mg}{min} \)
\end{choices}

\vspace{3mm}\question Die Leitfähigkeit von 120 ml Salzlösung wird zu 9,0 \( \frac{\mu S}{cm} \) bestimmt. Welche Leitfähigkeit wird gemessen, nachdem 60 ml destilliertes Wasser hinzugefügt wurden?

\begin{choices}
	\choice 12,0 \( \frac{\mu S}{cm} \)
	\choice 4,5 \( \frac{\mu S}{cm} \)
	\choice 0,0 \( \frac{\mu S}{cm} \)
	\choice 6,0 \( \frac{\mu S}{cm} \) (correct)
	\choice 3,0 \( \frac{\mu S}{cm} \)
\end{choices}

\vspace{3mm}\question Wie ändert sich der Diffusionsfluss \( J \) durch eine kreisförmige Membran, wenn deren Durchmesser sich verdreifacht?

\begin{choices}
	\choice \( J \) verneunfacht sich. (correct)
	\choice \( J \) nimmt um ein Drittel ab.
	\choice \( J \) verdreifacht sich.
	\choice \( J \) verdoppelt sich.
	\choice \( J \) ist unabhängig vom Durchmesser der Membran.
\end{choices}

\vspace{3mm}\question Von welchen Parametern ist der Diffusionsfluss durch ein Membran abhängig?A) TemperaturB) Fläche der MembranC) Größe der diffundierenden TeilchenD) Konzentration auf der Seite der höheren Konzentration

\begin{choices}
	\choice A und B
	\choice A, B und D
	\choice alle Antworten (correct)
	\choice B und C
	\choice keine Antwort
\end{choices}

\vspace{3mm}\question Was ist die SI-Einheit des Diffusionskoeffizienten?

\begin{choices}
	\choice \( \frac{s}{m^2} \)
	\choice \( \frac{kg}{s} \)
	\choice \( \frac{kg \cdot m^2}{s} \)
	\choice \( \frac{m^2}{s} \) (correct)
	\choice \( \frac{m^3}{s} \)
\end{choices}

\vspace{3mm}\end{questions}

\end{document}
