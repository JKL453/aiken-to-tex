\documentclass[11pt]{exam}

\usepackage{geometry}
\geometry{
a4paper,
total={185mm,257mm},
left=10mm,
top=25mm,
bottom=10mm
}

\begin{document}
\setlength{\voffset}{-0.5in}
\setlength{\headsep}{5pt}

\fbox{\fbox{\parbox{8cm}{\centering
\vspace{2mm}
Testat - Versuch E - Diffusion und Osmose
\vspace{2mm}
}}}
\hspace{2mm}
\makebox[0.25\textwidth]{Name:\enspace\hrulefill} \hspace{5mm}
\makebox[0.2\textwidth]{Datum:\enspace\hrulefill}
\vspace{4mm}

\begin{questions}

\question Wenn in zwei Kammern, die durch eine Membran von 50 mm Dicke getrennt sind, auf einer Seite eine Salzkonzentration von 30\( \frac{g}{cm^3} \) und auf der anderen Seite eine Konzentration von 5000 \( \frac{mg}{cm^3} \) herrscht, wie groß ist dann der Konzentrationsgradient über der Membran?

\begin{choices}
	\choice 5,0 \( \frac{mg}{cm^3} \)
	\choice 5,0 \( \frac{g}{cm^4} \)
	\choice 0,5 \( \frac{g}{cm^4} \)
	\choice 5,0 \( \frac{g}{cm^3\cdot mm} \)
	\choice 50,0 \( \frac{g}{cm^3\cdot mm} \)
\end{choices}

\vspace{3mm}\question Die Leitfähigkeit von 30 ml Salzlösung wird zu 15,0 \( \frac{\mu S}{cm} \) bestimmt. Diese Salzlösung wird nun mit destilliertem Wasser auf 90 ml aufgefüllt. Welche Leitfähigkeit wird nun gemessen?

\begin{choices}
	\choice 15,0 \( \frac{\mu S}{cm} \)
	\choice 45,0 \( \frac{\mu S}{cm} \)
	\choice 1,5 \( \frac{\mu S}{cm} \)
	\choice 60,0 \( \frac{\mu S}{cm} \)
	\choice 5,0 \( \frac{\mu S}{cm} \)
\end{choices}

\vspace{3mm}\question Der Diffusionsfluss durch eine rechteckige Membran mit den Seitenlängen 25 mm und 1 cm ist 1,5 \(\frac{mg}{s} \). Wie groß ist der Diffusionsfluss, wenn stattdessen eine Membran mit den Seitenlängen 5 cm und 10 mm verwendet wird?

\begin{choices}
	\choice \( J = 2,5 \frac{mg}{s} \)
	\choice \( J = 1,5 \frac{mg}{s} \)
	\choice \( J = 3,5 \frac{mg}{s} \)
	\choice \( J = 3,0 \frac{mg}{s} \)
	\choice \( J = 2,0 \frac{mg}{s} \)
\end{choices}

\vspace{3mm}\question Wo spielt Diffusion in der Natur keine Rolle?

\begin{choices}
	\choice Bildung des Zellpotentials
	\choice Gasaustausch bei der Lungenatmung
	\choice Bewegung von Amöben
	\choice Filtration in der Niere
	\choice Bildung von Ödemen
\end{choices}

\vspace{3mm}\question Von welchen Größen hängt der osmotische Druck eines gelösten Stoffes ab?

\begin{choices}
	\choice Konzentration des Stoffes und Temperatur
	\choice Dichte und Diffusionskoeffizient des Stoffes
	\choice Nur vom Van't-Hoff-Faktor des Stoffes
	\choice Gyrationsradius und Konzentration des Stoffes
	\choice Molare Masse und Ladung des Stoffes
\end{choices}

\vspace{3mm}\end{questions}

\end{document}
