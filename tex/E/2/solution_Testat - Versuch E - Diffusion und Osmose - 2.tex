\documentclass[11pt]{exam}

\usepackage{geometry}
\geometry{
a4paper,
total={185mm,257mm},
left=10mm,
top=25mm,
bottom=10mm
}

\begin{document}
\setlength{\voffset}{-0.5in}
\setlength{\headsep}{5pt}

\fbox{\fbox{\parbox{8cm}{\centering
\vspace{2mm}
Testat - Versuch E - Diffusion und Osmose - 2
\vspace{2mm}
}}}
\hspace{2mm}
\makebox[0.25\textwidth]{Name:\enspace\hrulefill} \hspace{5mm}
\makebox[0.2\textwidth]{Datum:\enspace\hrulefill}
\vspace{4mm}

\begin{questions}

\question In einer Doppelkammer, auf deren einer Seite eine Konzentration von 25 \(\frac{mg}{l} \) und auf der anderen von 10 \(  \frac{mg}{l} \) herrscht  und die durch eine Membran der Dicke 100 \(\mu \)m geteilt ist, herrscht ein Diffusionsfluss von 4  \( \frac{mg}{min}\). Wie groß ist der Diffusionsfluss, wenn die Konzentration auf der einen Seite von 10 \(  \frac{mg}{l} \)   auf 20 \(  \frac{mg}{l} \)  erhöht wird?

\begin{choices}
	\choice \( \frac{4}{3} \) \( \frac{mg}{min} \) (correct)
	\choice \( \frac{5}{4} \) \( \frac{mg}{min} \)
	\choice \( \frac{2}{3} \) \( \frac{mg}{min} \)
	\choice \( \frac{4}{5} \) \( \frac{mg}{min} \)
	\choice \( \frac{3}{4} \) \( \frac{mg}{min} \)
\end{choices}

\vspace{3mm}\question Die Leitfähigkeit von 20 ml Salzlösung wird zu 5,0 \( \frac{\mu S}{cm} \) bestimmt. Welches Volumen an destilliertem Wasser muss zugeführt werden, um eine Leitfähigkeit von 10,0 \( \frac{\mu S}{cm} \) zu messen?

\begin{choices}
	\choice 30 ml
	\choice 10 ml
	\choice keine Antwort richtig (correct)
	\choice 0 ml
	\choice 20 ml
\end{choices}

\vspace{3mm}\question Wie ändert sich der Diffusionsfluss \( J \) durch eine kreisförmige Membran, wenn deren Durchmesser sich verdoppelt und deren Dicke auf die Hälfte reduziert wird?

\begin{choices}
	\choice \( J \) wächst um das Vierfache.
	\choice \( J \) wächst auf das Doppelte.
	\choice \( J \) sinkt auf die Hälfte.
	\choice \( J \) wächst auf das Achtfache. (correct)
	\choice \( J \) quadriert sich.
\end{choices}

\vspace{3mm}\question Eine kolligative Eigenschaft ist eine Stoffeigenschaft, die ...

\begin{choices}
	\choice von der Temperatur abhängt.
	\choice von der Teilchenzahl und -größe abhängt.
	\choice nur von der Teilchenzahl abhängt. (correct)
	\choice von der Anzahl, Größe und Wertigkeit der Teilchen abhängt.
	\choice die Wechselwirkungen der Teilchen untereinander beschreibt.
\end{choices}

\vspace{3mm}\question Von welchen Größen hängt der osmotische Druck eines gelösten Stoffes ab?

\begin{choices}
	\choice Dichte und Diffusionskoeffizient des Stoffes
	\choice Konzentration des Stoffes und Temperatur (correct)
	\choice Nur vom Van't-Hoff-Faktor des Stoffes
	\choice Molare Masse und Ladung des Stoffes
	\choice Gyrationsradius und Konzentration des Stoffes
\end{choices}

\vspace{3mm}\end{questions}

\end{document}
