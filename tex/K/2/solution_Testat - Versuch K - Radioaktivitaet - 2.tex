\documentclass[11pt]{exam}

\usepackage{geometry}
\geometry{
a4paper,
total={185mm,257mm},
left=10mm,
top=25mm,
bottom=10mm
}

\begin{document}
\setlength{\voffset}{-0.5in}
\setlength{\headsep}{5pt}

\fbox{\fbox{\parbox{8cm}{\centering
\vspace{2mm}
Testat - Versuch K - Radioaktivitaet - 2
\vspace{2mm}
}}}
\hspace{2mm}
\makebox[0.25\textwidth]{Name:\enspace\hrulefill} \hspace{5mm}
\makebox[0.2\textwidth]{Datum:\enspace\hrulefill}
\vspace{4mm}

\begin{questions}

\question Die Aktivität einer radioaktiven Substanz wird in Becquerel angegeben. Wie setzt sich diese Größe aus den  SI-Basiseinheiten zusammen?

\begin{choices}
	\choice \( \text{s}^{-1} \) (correct)
	\choice s
	\choice \( \frac{\text{J}}{\text{s}} \)
	\choice \( \text{kg}\cdot\frac{\text{m}}{\text{s}^2} \)
	\choice \( \frac{\text{J}}{\text{kg}} \)
\end{choices}

\vspace{3mm}\question Welche Aussage ist richtig?

\begin{choices}
	\choice Beim \( \beta \)-Zerfall wandelt sich ein Neutron in ein Proton um. (correct)
	\choice Bei der \( \gamma \)-Umwandlung entsteht ein Elektron als freies Teilchen.
	\choice Beim \( \alpha^- \)-Zerfall entsteht ein negativ geladener Heliumkern als freies Teilchen.
	\choice Bei der \( \gamma \)-Umwandlung wird die freiwerdende Energie auf zwei Teilchen, ein Elektron und ein Antineutrino, aufgeteilt.
	\choice Beim \( \beta \)-Zerfall entsteht kein freies Teilchen sondern eine elektromagnetische Welle, die Betawelle.
\end{choices}

\vspace{3mm}\question Die Halbwertsdicke eines Absorbers beträgt für radioaktives Barium-137m ca. 5 mm. Wieviel Prozent der Strahlung ist nach 12 mm übrig?

\begin{choices}
	\choice 38 %
	\choice 40 %
	\choice 19 % (correct)
	\choice 12,5 %
	\choice 25 %
\end{choices}

\vspace{3mm}\question Welche Aussage ist richtig?

\begin{choices}
	\choice Beim \( \beta \)-Zerfall wandelt sich ein Neutron in ein Proton um. (correct)
	\choice Beim \( \beta \)-Zerfall entsteht kein freies Teilchen sondern eine elektromagnetische Welle, die Betawelle.
	\choice Bei der \( \gamma \)-Umwandlung wird die freiwerdende Energie auf zwei Teilchen, ein Elektron und ein Antineutrino, aufgeteilt.
	\choice Beim \( \alpha^- \)-Zerfall entsteht ein negativ geladener Heliumkern als freies Teilchen.
	\choice Bei der \( \gamma \)-Umwandlung entsteht ein Elektron als freies Teilchen.
\end{choices}

\vspace{3mm}\question Das Schwächungsgesetz beschreibt den Zusammenhang zwischen Intensität der Strahlung und Absorberdicke. Der Zusammenhang zwischen diesen Größen ist ...

\begin{choices}
	\choice quadratisch.
	\choice exponentiell. (correct)
	\choice proportional.
	\choice antiproportional.
	\choice keine der Antworten ist richtig.
\end{choices}

\vspace{3mm}\end{questions}

\end{document}
