\documentclass[11pt]{exam}

\usepackage{geometry}
\geometry{
a4paper,
total={185mm,257mm},
left=10mm,
top=25mm,
bottom=10mm
}

\begin{document}
\setlength{\voffset}{-0.5in}
\setlength{\headsep}{5pt}

\fbox{\fbox{\parbox{8cm}{\centering
\vspace{2mm}
Testat - Versuch K - Radioaktivitaet - 1
\vspace{2mm}
}}}
\hspace{2mm}
\makebox[0.25\textwidth]{Name:\enspace\hrulefill} \hspace{5mm}
\makebox[0.2\textwidth]{Datum:\enspace\hrulefill}
\vspace{4mm}

\begin{questions}

\question Die Aktivität einer radioaktiven Substanz wird in Becquerel angegeben. Wie setzt sich diese Größe aus den  SI-Basiseinheiten zusammen?

\begin{choices}
	\choice s
	\choice \( \frac{\text{J}}{\text{s}} \)
	\choice \( \frac{\text{J}}{\text{kg}} \)
	\choice \( \text{kg}\cdot\frac{\text{m}}{\text{s}^2} \)
	\choice \( \text{s}^{-1} \)
\end{choices}

\vspace{3mm}\question Welche Zuordnung von Zerfallsart und frei werdendem Teilchen ist falsch?

\begin{choices}
	\choice \( \beta^- \)-Zerfall - Elektron
	\choice \( \gamma \)-Umwandlung - elektromagnetische Welle
	\choice \( \beta^- \)-Zerfall - Antineutrino
	\choice \( \beta^+ \)-Zerfall - Proton
	\choice \( \alpha \)-Zerfall - Heliumkern
\end{choices}

\vspace{3mm}\question Die Halbwertszeit von Jod-131 beträgt ca. 8 Tage. Die Anfangsaktivität beträgt 500 Bq. Wie groß ist die Aktivität nach 10 Tagen?

\begin{choices}
	\choice 500 Bq
	\choice 210 Bq
	\choice 250 Bq
	\choice 1290 Bq
	\choice 125 Bq
\end{choices}

\vspace{3mm}\question Was hat keinen Einfluss auf die gemessene Zählrate?

\begin{choices}
	\choice Aktivität der Quelle
	\choice Absorbermaterial
	\choice Temperatur
	\choice Absorberdicke
	\choice Abstand zur Quelle
\end{choices}

\vspace{3mm}\question Was ist keine Wechselwirkung von radioaktiver Strahlung mit Materie?

\begin{choices}
	\choice Piezo-Effekt
	\choice Photo-Effekt
	\choice Compton-Effekt
	\choice Alle sind richtig
	\choice Stoßionisation
\end{choices}

\vspace{3mm}\end{questions}

\end{document}
