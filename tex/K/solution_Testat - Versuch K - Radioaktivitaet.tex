\documentclass[11pt]{exam}

\usepackage{geometry}
\geometry{
a4paper,
total={185mm,257mm},
left=10mm,
top=25mm,
bottom=10mm
}

\begin{document}
\setlength{\voffset}{-0.5in}
\setlength{\headsep}{5pt}

\fbox{\fbox{\parbox{8cm}{\centering
\vspace{2mm}
Testat - Versuch K - Radioaktivitaet
\vspace{2mm}
}}}
\hspace{2mm}
\makebox[0.25\textwidth]{Name:\enspace\hrulefill} \hspace{5mm}
\makebox[0.2\textwidth]{Datum:\enspace\hrulefill}
\vspace{4mm}

\begin{questions}

\question Die mittlere Lebensdauer \( \tau \) eines radioaktiven Präparates ist definiert als:

\begin{choices}
	\choice \( \tau = T_{1/2} \)
	\choice \( \tau = 1/T_{1/2} \)
	\choice Zerfälle pro Zeit
	\choice \( \tau = 1/\lambda \) (correct)
	\choice \( \tau = \ln 2 / \lambda \)
\end{choices}

\vspace{3mm}\question Welche Zuordnung von Zerfallsart und frei werdendem Teilchen ist falsch?

\begin{choices}
	\choice \( \gamma \)-Umwandlung - elektromagnetische Welle
	\choice \( \beta^- \)-Zerfall - Antineutrino
	\choice \( \beta^- \)-Zerfall - Elektron
	\choice \( \beta^+ \)-Zerfall - Proton (correct)
	\choice \( \alpha \)-Zerfall - Heliumkern
\end{choices}

\vspace{3mm}\question Die Halbwertszeit von Jod-131 beträgt ca. 8 Tage. Die Anfangsaktivität beträgt 500 Bq. Wie groß ist die Aktivität nach 10 Tagen?

\begin{choices}
	\choice 250 Bq
	\choice 210 Bq (correct)
	\choice 125 Bq
	\choice 500 Bq
	\choice 1290 Bq
\end{choices}

\vspace{3mm}\question Welche Aussage ist richtig?

\begin{choices}
	\choice Bei der \( \gamma \)-Umwandlung entsteht ein Elektron als freies Teilchen.
	\choice Beim \( \beta \)-Zerfall entsteht kein freies Teilchen sondern eine elektromagnetische Welle, die Betawelle.
	\choice Beim \( \alpha^- \)-Zerfall entsteht ein negativ geladener Heliumkern als freies Teilchen.
	\choice Bei der \( \gamma \)-Umwandlung wird die freiwerdende Energie auf zwei Teilchen, ein Elektron und ein Antineutrino, aufgeteilt.
	\choice Beim \( \beta \)-Zerfall wandelt sich ein Neutron in ein Proton um. (correct)
\end{choices}

\vspace{3mm}\question Welche Strahlungsart hat die kleinste Reichweite in Materie (bei gleicher Energie)?

\begin{choices}
	\choice Röntgenstrahlung
	\choice gamma-Strahlung
	\choice \( \beta^+ \)-Strahlung
	\choice \( \beta^- \)-Strahlung
	\choice alpha-Strahlung (correct)
\end{choices}

\vspace{3mm}\end{questions}

\end{document}
