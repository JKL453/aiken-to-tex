\documentclass[11pt]{exam}

\usepackage{geometry}
\geometry{
a4paper,
total={185mm,257mm},
left=10mm,
top=25mm,
bottom=10mm
}

\begin{document}
\setlength{\voffset}{-0.5in}
\setlength{\headsep}{5pt}

\fbox{\fbox{\parbox{8cm}{\centering
\vspace{2mm}
Testat - Versuch K - Radioaktivitaet
\vspace{2mm}
}}}
\hspace{2mm}
\makebox[0.25\textwidth]{Name:\enspace\hrulefill} \hspace{5mm}
\makebox[0.2\textwidth]{Datum:\enspace\hrulefill}
\vspace{4mm}

\begin{questions}

\question Wie lautet die Einheit der physikalischen Größe Aktivität?

\begin{choices}
	\choice Becquerel (Bq)
	\choice Gray (Gy)
	\choice Zeit (t)
	\choice Joule (J)
	\choice Meter (m)
\end{choices}

\vspace{3mm}\question Welche Aussage ist falsch?

\begin{choices}
	\choice  Beim Photoeffekt entsteht Sekundärstrahlung.
	\choice Elektronen verlieren ihre Energie nur durch Stoßionisationen.
	\choice Beim Comptoneffekt entsteht Sekundärstrahlung.
	\choice Stoßionisation kommt z.B. in einem Geiger-Müller-Zählrohr vor.
	\choice Gammastrahlung kann z.B. durch den Photoeffekt absorbiert werden.
\end{choices}

\vspace{3mm}\question Die Halbwertszeit von Jod-131 beträgt ca. 8 Tage. Die Anfangsaktivität beträgt 500 Bq. Wie groß ist die Aktivität nach 10 Tagen?

\begin{choices}
	\choice 250 Bq
	\choice 500 Bq
	\choice 125 Bq
	\choice 210 Bq
	\choice 1290 Bq
\end{choices}

\vspace{3mm}\question Welche Aussage ist falsch?

\begin{choices}
	\choice gamma-Strahlung ist immer ungeladen.
	\choice Teilchenstrahlung kann als Folge eines radioaktiven Zerfalls entstehen.
	\choice Beim \( \beta^- \)-Zerfall wird u.a. eine Elektron emittiert.
	\choice Elektromagnetische Wellenstrahlung kommt in der Natur nicht vor, sondern kann nur künstlich erzeugt werden (z.B. Röntgenröhre).
	\choice alpha-Strahlung ist immer positiv geladen.
\end{choices}

\vspace{3mm}\question Das Zerfallsgesetz beschreibt den Zusammenhang zwischen der Aktivität einer radioaktiven Quelle und der Zeit. Der Zusammenhang zwischen diesen Größen ist ...

\begin{choices}
	\choice quadratisch.
	\choice exponentiell.
	\choice proportional.
	\choice keine der Antworten ist richtig.
	\choice antiproportional.
\end{choices}

\vspace{3mm}\end{questions}

\end{document}
