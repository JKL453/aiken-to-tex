\documentclass[11pt]{exam}

\usepackage{geometry}
\geometry{
a4paper,
total={185mm,257mm},
left=10mm,
top=25mm,
bottom=10mm
}

\begin{document}
\setlength{\voffset}{-0.5in}
\setlength{\headsep}{5pt}

\fbox{\fbox{\parbox{8cm}{\centering
\vspace{2mm}
Testat - Versuch H - Spektralphotometer
\vspace{2mm}
}}}
\hspace{2mm}
\makebox[0.25\textwidth]{Name:\enspace\hrulefill} \hspace{5mm}
\makebox[0.2\textwidth]{Datum:\enspace\hrulefill}
\vspace{4mm}

\begin{questions}

\question Wie heißt der an die langwellige Grenze des sichtbaren Spektrums grenzende Spektralbereich?

\begin{choices}
	\choice Ultraviolett
	\choice Infrarot (correct)
	\choice Mikrowelle
	\choice Röntgenstrahlung
	\choice Langwelle
\end{choices}

\vspace{3mm}\question Ordnen Sie die folgenden elektromagnetischen Wellen aufsteigend nach ihrer Frequenz:weiche Röntgenstrahlung: \( \nu_1 \)Gammastrahlung: \( \nu_2 \)sichtbares Licht: \( \nu_3 \)Radiokurzwellen: \( \nu_4 \)

\begin{choices}
	\choice \( \nu_3 \) < \( \nu_4 \) < \( \nu_2 \) < \( \nu_1 \)
	\choice \( \nu_2 \) < \( \nu_1 \) < \( \nu_3 \) < \( \nu_4 \)
	\choice \( \nu_1 \) < \( \nu_2 \) < \( \nu_3 \) < \( \nu_4 \)
	\choice \( \nu_4 \) < \( \nu_1 \) < \( \nu_2 \) < \( \nu_3 \)
	\choice \( \nu_4 \) < \( \nu_3 \) < \( \nu_1 \) < \( \nu_2 \) (correct)
\end{choices}

\vspace{3mm}\question Farbstofflösung transmittiert 1 % des auftreffenden monochromatischen Lichts. Die Extinktion \( E=lg \frac{I_0}{I} \) dieser Lösung beträgt dann

\begin{choices}
	\choice 0,01
	\choice 2 (correct)
	\choice 0,99
	\choice 99
	\choice 1
\end{choices}

\vspace{3mm}\question Ordnen Sie die folgenden elektromagnetischen Wellen aufsteigend nach ihrer Frequenz:weiche Röntgenstrahlung: \( \nu_1 \)Gammastrahlung: \( \nu_2 \)sichtbares Licht: \( \nu_3 \)Radiokurzwellen: \( \nu_4 \)

\begin{choices}
	\choice \( \nu_1 \) < \( \nu_2 \) < \( \nu_3 \) < \( \nu_4 \)
	\choice \( \nu_4 \) < \( \nu_1 \) < \( \nu_2 \) < \( \nu_3 \)
	\choice \( \nu_2 \) < \( \nu_1 \) < \( \nu_3 \) < \( \nu_4 \)
	\choice \( \nu_4 \) < \( \nu_3 \) < \( \nu_1 \) < \( \nu_2 \) (correct)
	\choice \( \nu_3 \) < \( \nu_4 \) < \( \nu_2 \) < \( \nu_1 \)
\end{choices}

\vspace{3mm}\question Elektromagnetische Wellen im Bereich des sichtbaren Lichts ...

\begin{choices}
	\choice haben eine höhere Frequenz als IR-Strahlung. (correct)
	\choice haben die gleiche Wellenlänge wie Gamma-Strahlung.
	\choice haben eine kürzere Wellenlänge als Röntgenstrahlung.
	\choice haben eine längere Wellenlänge als Radiowellen.
	\choice haben eine höhere Frequenz als UV-Strahlung.
\end{choices}

\vspace{3mm}\end{questions}

\end{document}
