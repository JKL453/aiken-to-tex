\documentclass[11pt]{exam}

\usepackage{geometry}
\geometry{
a4paper,
total={185mm,257mm},
left=10mm,
top=25mm,
bottom=10mm
}

\begin{document}
\setlength{\voffset}{-0.5in}
\setlength{\headsep}{5pt}

\fbox{\fbox{\parbox{8cm}{\centering
\vspace{2mm}
Testat - Versuch H - Spektralphotometer
\vspace{2mm}
}}}
\hspace{2mm}
\makebox[0.25\textwidth]{Name:\enspace\hrulefill} \hspace{5mm}
\makebox[0.2\textwidth]{Datum:\enspace\hrulefill}
\vspace{4mm}

\begin{questions}

\question Bei einer photometrischen Konzentrationsbestimmung in der Labormedizin wird die Lichtschwächung gemessen, die von der zu untersuchenden Substanz in der Messküvette bewirkt wird. Die in der Flüssigkeit eingestrahlte Lichtintensität ist \( I_0 \) und die die Flüssigkeit verlassende Lichtintensität ist \(I\). Der Leer-Wert sei vernachlässigbar oder bereits abgezogen. Welche Proportionalität zur Substanz besteht für verdünnte Lösungen und monochromatische Messstrahlung typischerweisenach Lambert Beer?

\begin{choices}
	\choice \( e^{( \frac{I_0}{I})} \sim c \)
	\choice \( \frac{I}{I_0} \sim c \)
	\choice \( \frac{I_0}{I} \sim c \)
	\choice \( ln ( \frac{I_0}{I}) \sim c \)
	\choice \( 1/( \frac{I_0}{I}) \sim c \)
\end{choices}

\vspace{3mm}\question Eine Farbstofflösung absorbiert 90 % des auftreffenden monochromatischen Lichts. Die Extinktion \( E=lg \frac{I_0}{I} \) dieser Lösung beträgt dann

\begin{choices}
	\choice 1
	\choice 2
	\choice 0,01
	\choice 0,99
	\choice 99
\end{choices}

\vspace{3mm}\question Die dekadische Extinktion E ist (ohne Reflexion) der Zehnerlogarithmus des Quotienten aus eingestrahlter und durchgelassener Intensität. Wie viel Prozent des einfallenden monochromatischen Lichtes wird von einer Lösung mit \(E = 1\) absorbiert?

\begin{choices}
	\choice 10 %
	\choice 100 %
	\choice 90 %
	\choice 1 %
	\choice 5 %
\end{choices}

\vspace{3mm}\question Ordnen Sie die folgenden elektromagnetischen Wellen aufsteigend nach ihrer Wellenlänge:weiche Röntgenstrahlung: \( \lambda_1 \)Gammastrahlung: \( \lambda_2 \)sichtbares Licht: \( \lambda_3 \)Radiokurzwellen: \( \lambda_4 \)

\begin{choices}
	\choice \( \lambda_4 < \lambda_1 < \lambda_2 < \lambda_3 \)
	\choice \( \lambda_2 < \lambda_1 < \lambda_3 < \lambda_4 \)
	\choice \( \lambda_4 < \lambda_3 < \lambda_1 < \lambda_2 \)
	\choice \( \lambda_3 < \lambda_4 < \lambda_2 < \lambda_1 \)
	\choice \( \lambda_1 < \lambda_2 < \lambda_3 < \lambda_4 \)
\end{choices}

\vspace{3mm}\question Eine Farbstoffl¨osung absorbiert 99 % des auftreffenden monochromatischen Lichts. Die Extinktion \( E=lg \frac{I_0}{I} \) dieser Lösung beträgt dann

\begin{choices}
	\choice 0,99
	\choice 1
	\choice 0,01
	\choice 99
	\choice 2
\end{choices}

\vspace{3mm}\end{questions}

\end{document}
