\documentclass[11pt]{exam}

\usepackage{amsmath}
\usepackage{graphicx}
\usepackage{geometry}
\usepackage{etoolbox}
\BeforeBeginEnvironment{choices}{\par\nopagebreak\minipage{\linewidth}}
\AfterEndEnvironment{choices}{\endminipage}
\geometry{
a4paper,
total={185mm,257mm},
left=10mm,
top=25mm,
bottom=10mm
}

\begin{document}
\setlength{\voffset}{-0.5in}
\setlength{\headsep}{5pt}

\fbox{\fbox{\parbox{8cm}{\centering
\vspace{2mm}
Testat - Versuch H - Spektralphotometer - 3
\vspace{2mm}
}}}
\hspace{2mm}
\makebox[0.25\textwidth]{Name:\enspace\hrulefill} \hspace{5mm}
\makebox[0.2\textwidth]{Datum:\enspace\hrulefill}
\vspace{4mm}

\begin{questions}

\question Bei einer photometrischen Konzentrationsbestimmung in der Labormedizin wird die Lichtschwächung gemessen, die von der zu untersuchenden Substanz in der Messküvette bewirkt wird. Die in der Flüssigkeit eingestrahlte Lichtintensität ist \( I_0 \) und die die Flüssigkeit verlassende Lichtintensität ist \(I\). Der Leer-Wert sei vernachlässigbar oder bereits abgezogen. Welche Proportionalität zur Substanz besteht für verdünnte Lösungen und monochromatische Messstrahlung typischerweisenach Lambert Beer?

\begin{choices}
	\choice \( e^{( \frac{I_0}{I})} \sim c \)
	\choice \( \frac{I_0}{I} \sim c \)
	\choice \( \frac{I}{I_0} \sim c \)
	\choice \( 1/( \frac{I_0}{I}) \sim c \)
	\choice \( ln ( \frac{I_0}{I}) \sim c \)
\end{choices}

\vspace{3mm}\question Die Lichtgeschwindigkeit im Vakuum betr¨agt ungefähr:

\begin{choices}
	\choice \( 3 \cdot10^5 m/s \)
	\choice \( 3 \cdot10^5 km/h \)
	\choice Keine der Aussagen trifft zu.
	\choice \( 3 \cdot10^5 km/s \)
	\choice \( 3 \cdot10^5 m/h \)
\end{choices}

\vspace{3mm}\question Eine Farbstofflösung transmittiert 10 % des auftreffenden monochromatischen Lichts. Die Extinktion \( E=lg \frac{I_0}{I} \) dieser Lösung beträgt dann

\begin{choices}
	\choice 0,01
	\choice 0,99
	\choice 99
	\choice 1
	\choice 2
\end{choices}

\vspace{3mm}\question Wie heißt der an die kurzwellige Grenze des sichtbaren Spektrums grenzende Spektralbereich?

\begin{choices}
	\choice Röntgenstrahlung
	\choice Langwelle
	\choice Infrarot
	\choice Mikrowelle
	\choice Ultraviolett
\end{choices}

\vspace{3mm}\question Wenn \( \lambda_1 > \lambda_2 \), dann gilt für  \( \lambda_1 \)

\begin{choices}
	\choice Kleinere Frequenz, höhere Energie als \( \lambda_2 \).
	\choice Höhere Frequenz, gleiche Energie wie \( \lambda_2 \).
	\choice Höhere Frequenz, kleinere Energie als \( \lambda_2 \).
	\choice Kleinere Frequenz, gleiche Energie wie \( \lambda_2 \).
	\choice Kleinere Frequenz, kleinere Energie als \( \lambda_2 \).
\end{choices}

\vspace{3mm}\end{questions}

\end{document}
