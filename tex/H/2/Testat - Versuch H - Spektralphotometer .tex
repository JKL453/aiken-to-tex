\documentclass[11pt]{exam}

\usepackage{amsmath}
\usepackage{graphicx}
\usepackage{geometry}
\usepackage{etoolbox}
\BeforeBeginEnvironment{choices}{\par\nopagebreak\minipage{\linewidth}}
\AfterEndEnvironment{choices}{\endminipage}
\geometry{
a4paper,
total={185mm,257mm},
left=10mm,
top=25mm,
bottom=10mm
}

\begin{document}
\setlength{\voffset}{-0.5in}
\setlength{\headsep}{5pt}

\fbox{\fbox{\parbox{8cm}{\centering
\vspace{2mm}
Testat - Versuch H - Spektralphotometer 
\vspace{2mm}
}}}
\hspace{2mm}
\makebox[0.25\textwidth]{Name:\enspace\hrulefill} \hspace{5mm}
\makebox[0.2\textwidth]{Datum:\enspace\hrulefill}
\vspace{4mm}

\begin{questions}

\question Das Lambert-Beersche Gesetz besagt, dass die Transmission ...

\begin{choices}
	\choice mit zunehmender Konzentration der Lösung größer wird.
	\choice nicht von der Zusammensetzung der Lösung abhängt.
	\choice exponentiell mit steigender Konzentration der Lösung abnimmt.
	\choice unabhängig von der Wellenlänge des Lichtes ist.
	\choice linear von der Schichtdicke d abhängt.
\end{choices}

\vspace{3mm}\question Die Lichtgeschwindigkeit im Vakuum betr¨agt ungefähr:

\begin{choices}
	\choice \( 3 \cdot10^5 km/h \)
	\choice \( 3 \cdot10^5 m/h \)
	\choice \( 3 \cdot10^5 m/s \)
	\choice \( 3 \cdot10^5 km/s \)
	\choice Keine der Aussagen trifft zu.
\end{choices}

\vspace{3mm}\question Farbstofflösung transmittiert 1 % des auftreffenden monochromatischen Lichts. Die Extinktion \( E=lg \frac{I_0}{I} \) dieser Lösung beträgt dann

\begin{choices}
	\choice 99
	\choice 0,99
	\choice 0,01
	\choice 2
	\choice 1
\end{choices}

\vspace{3mm}\question Wie heißt der an die kurzwellige Grenze des sichtbaren Spektrums grenzende Spektralbereich?

\begin{choices}
	\choice Mikrowelle
	\choice Langwelle
	\choice Röntgenstrahlung
	\choice Infrarot
	\choice Ultraviolett
\end{choices}

\vspace{3mm}\question Welcher der nachfolgenden Wellenlängenbereiche entspricht am ehesten dem Bereich des sichtbarenLichtes?

\begin{choices}
	\choice \( 200 - 500 nm \)
	\choice \( 350 - 750 nm \)
	\choice \( 500 - 900 nm \)
	\choice \( 300 - 600 nm \)
	\choice \( 100 - 400 nm \)
\end{choices}

\vspace{3mm}\end{questions}

\end{document}
