\documentclass[11pt]{exam}

\usepackage{geometry}
\geometry{
a4paper,
total={185mm,257mm},
left=10mm,
top=25mm,
bottom=10mm
}

\begin{document}
\setlength{\voffset}{-0.5in}
\setlength{\headsep}{5pt}

\fbox{\fbox{\parbox{8cm}{\centering
\vspace{2mm}
Testat - Versuch H - Spektralphotometer - 2
\vspace{2mm}
}}}
\hspace{2mm}
\makebox[0.25\textwidth]{Name:\enspace\hrulefill} \hspace{5mm}
\makebox[0.2\textwidth]{Datum:\enspace\hrulefill}
\vspace{4mm}

\begin{questions}

\question Das Lambert-Beersche Gesetz besagt, dass die Transmission ...

\begin{choices}
	\choice exponentiell mit steigender Konzentration der Lösung abnimmt. (correct)
	\choice unabhängig von der Wellenlänge des Lichtes ist.
	\choice mit zunehmender Konzentration der Lösung größer wird.
	\choice linear von der Schichtdicke d abhängt.
	\choice nicht von der Zusammensetzung der Lösung abhängt.
\end{choices}

\vspace{3mm}\question Welche Kurve der Abbildung gibt den Verlauf der Extinktion in Abhängigkeit von der Materialdickerichtig wieder?

\begin{choices}
	\choice D
	\choice A
	\choice C (correct)
	\choice E
	\choice B
\end{choices}

\vspace{3mm}\question Welche Kurve der Abbildung gibt den Verlauf der Extinktion in Abhängigkeit von der Konzentrationrichtig wieder?

\begin{choices}
	\choice A
	\choice D
	\choice E
	\choice B
	\choice C (correct)
\end{choices}

\vspace{3mm}\question Eine Farbstoffl¨osung absorbiert 99 % des auftreffenden monochromatischen Lichts. Die Extinktion \( E=lg \frac{I_0}{I} \) dieser Lösung beträgt dann

\begin{choices}
	\choice 1
	\choice 0,99
	\choice 2 (correct)
	\choice 0,01
	\choice 99
\end{choices}

\vspace{3mm}\question Ordnen Sie die folgenden elektromagnetischen Wellen aufsteigend nach ihrer Wellenlänge:weiche Röntgenstrahlung: \( \lambda_1 \)Gammastrahlung: \( \lambda_2 \)sichtbares Licht: \( \lambda_3 \)Radiokurzwellen: \( \lambda_4 \)

\begin{choices}
	\choice \( \lambda_3 < \lambda_4 < \lambda_2 < \lambda_1 \)
	\choice \( \lambda_4 < \lambda_3 < \lambda_1 < \lambda_2 \)
	\choice \( \lambda_1 < \lambda_2 < \lambda_3 < \lambda_4 \)
	\choice \( \lambda_2 < \lambda_1 < \lambda_3 < \lambda_4 \) (correct)
	\choice \( \lambda_4 < \lambda_1 < \lambda_2 < \lambda_3 \)
\end{choices}

\vspace{3mm}\end{questions}

\end{document}
