\documentclass[11pt]{exam}

\usepackage{amsmath}
\usepackage{graphicx}
\usepackage{geometry}
\usepackage{etoolbox}
\BeforeBeginEnvironment{choices}{\par\nopagebreak\minipage{\linewidth}}
\AfterEndEnvironment{choices}{\endminipage}
\geometry{
a4paper,
total={185mm,257mm},
left=10mm,
top=25mm,
bottom=10mm
}

\begin{document}
\setlength{\voffset}{-0.5in}
\setlength{\headsep}{5pt}

\fbox{\fbox{\parbox{8cm}{\centering
\vspace{2mm}
Testat - Versuch H - Spektralphotometer - 1
\vspace{2mm}
}}}
\hspace{2mm}
\makebox[0.25\textwidth]{Name:\enspace\hrulefill} \hspace{5mm}
\makebox[0.2\textwidth]{Datum:\enspace\hrulefill}
\vspace{4mm}

\begin{questions}

\question Das Lambert-Beersche Gesetz besagt, dass die Extinktion ...

\begin{choices}
	\choice exponentiell mit steigender Konzentration der Lösung abnimmt.
	\choice nicht von der Zusammensetzung der Lösung abhängt.
	\choice mit zunehmender Konzentration der Lösung kleiner wird.
	\choice unabhängig von der Wellenlänge des Lichtes ist.
	\choice linear von der Schichtdicke d abhängt. (correct)
\end{choices}

\vspace{3mm}\question Die Lichtgeschwindigkeit im Vakuum betr¨agt ungefähr:

\begin{choices}
	\choice Keine der Aussagen trifft zu.
	\choice \( 3 \cdot10^5 km/s \) (correct)
	\choice \( 3 \cdot10^5 m/s \)
	\choice \( 3 \cdot10^5 m/h \)
	\choice \( 3 \cdot10^5 km/h \)
\end{choices}

\vspace{3mm}\question Eine Farbstoffl¨osung absorbiert 99 % des auftreffenden monochromatischen Lichts. Die Extinktion \( E=lg \frac{I_0}{I} \) dieser Lösung beträgt dann

\begin{choices}
	\choice 99
	\choice 0,01
	\choice 1
	\choice 0,99
	\choice 2 (correct)
\end{choices}

\vspace{3mm}\question Wie heißt der an die kurzwellige Grenze des sichtbaren Spektrums grenzende Spektralbereich?

\begin{choices}
	\choice Röntgenstrahlung
	\choice Langwelle
	\choice Infrarot
	\choice Ultraviolett (correct)
	\choice Mikrowelle
\end{choices}

\vspace{3mm}\question Ordnen Sie die folgenden elektromagnetischen Wellen aufsteigend nach ihrer Frequenz:weiche Röntgenstrahlung: \( \nu_1 \)Gammastrahlung: \( \nu_2 \)sichtbares Licht: \( \nu_3 \)Radiokurzwellen: \( \nu_4 \)

\begin{choices}
	\choice \( \nu_3 \) < \( \nu_4 \) < \( \nu_2 \) < \( \nu_1 \)
	\choice \( \nu_1 \) < \( \nu_2 \) < \( \nu_3 \) < \( \nu_4 \)
	\choice \( \nu_2 \) < \( \nu_1 \) < \( \nu_3 \) < \( \nu_4 \)
	\choice \( \nu_4 \) < \( \nu_1 \) < \( \nu_2 \) < \( \nu_3 \)
	\choice \( \nu_4 \) < \( \nu_3 \) < \( \nu_1 \) < \( \nu_2 \) (correct)
\end{choices}

\vspace{3mm}\end{questions}

\end{document}
