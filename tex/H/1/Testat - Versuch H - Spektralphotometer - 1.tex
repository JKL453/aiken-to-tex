\documentclass[11pt]{exam}

\usepackage{geometry}
\geometry{
a4paper,
total={185mm,257mm},
left=10mm,
top=25mm,
bottom=10mm
}

\begin{document}
\setlength{\voffset}{-0.5in}
\setlength{\headsep}{5pt}

\fbox{\fbox{\parbox{8cm}{\centering
\vspace{2mm}
Testat - Versuch H - Spektralphotometer - 1
\vspace{2mm}
}}}
\hspace{2mm}
\makebox[0.25\textwidth]{Name:\enspace\hrulefill} \hspace{5mm}
\makebox[0.2\textwidth]{Datum:\enspace\hrulefill}
\vspace{4mm}

\begin{questions}

\question Die Lichtgeschwindigkeit im Vakuum betr¨agt ungefähr:

\begin{choices}
	\choice Keine der Aussagen trifft zu.
	\choice \( 3 \cdot10^5 km/s \)
	\choice \( 3 \cdot10^5 m/h \)
	\choice \( 3 \cdot10^5 m/s \)
	\choice \( 3 \cdot10^5 km/h \)
\end{choices}

\vspace{3mm}\question Das Lambert-Beersche Gesetz besagt, dass die Transmission ...

\begin{choices}
	\choice linear von der Schichtdicke d abhängt.
	\choice exponentiell mit steigender Konzentration der Lösung abnimmt.
	\choice mit zunehmender Konzentration der Lösung größer wird.
	\choice unabhängig von der Wellenlänge des Lichtes ist.
	\choice nicht von der Zusammensetzung der Lösung abhängt.
\end{choices}

\vspace{3mm}\question Ordnen Sie die folgenden elektromagnetischen Wellen aufsteigend nach ihrer Wellenlänge:weiche Röntgenstrahlung: \( \lambda_1 \)Gammastrahlung: \( \lambda_2 \)sichtbares Licht: \( \lambda_3 \)Radiokurzwellen: \( \lambda_4 \)

\begin{choices}
	\choice \( \lambda_1 < \lambda_2 < \lambda_3 < \lambda_4 \)
	\choice \( \lambda_2 < \lambda_1 < \lambda_3 < \lambda_4 \)
	\choice \( \lambda_3 < \lambda_4 < \lambda_2 < \lambda_1 \)
	\choice \( \lambda_4 < \lambda_3 < \lambda_1 < \lambda_2 \)
	\choice \( \lambda_4 < \lambda_1 < \lambda_2 < \lambda_3 \)
\end{choices}

\vspace{3mm}\question Welche Kurve der Abbildung gibt den Verlauf der Extinktion in Abhängigkeit von der Konzentrationrichtig wieder?

\begin{choices}
	\choice E
	\choice D
	\choice B
	\choice A
	\choice C
\end{choices}

\vspace{3mm}\question Das Lambert-Beersche Gesetz besagt, dass die Extinktion ...

\begin{choices}
	\choice linear von der Schichtdicke d abhängt.
	\choice unabhängig von der Wellenlänge des Lichtes ist.
	\choice mit zunehmender Konzentration der Lösung kleiner wird.
	\choice exponentiell mit steigender Konzentration der Lösung abnimmt.
	\choice nicht von der Zusammensetzung der Lösung abhängt.
\end{choices}

\vspace{3mm}\end{questions}

\end{document}
