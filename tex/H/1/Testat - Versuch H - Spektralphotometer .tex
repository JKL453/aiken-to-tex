\documentclass[11pt]{exam}

\usepackage{amsmath}
\usepackage{graphicx}
\usepackage{geometry}
\usepackage{etoolbox}
\BeforeBeginEnvironment{choices}{\par\nopagebreak\minipage{\linewidth}}
\AfterEndEnvironment{choices}{\endminipage}
\geometry{
a4paper,
total={185mm,257mm},
left=10mm,
top=25mm,
bottom=10mm
}

\begin{document}
\setlength{\voffset}{-0.5in}
\setlength{\headsep}{5pt}

\fbox{\fbox{\parbox{8cm}{\centering
\vspace{2mm}
Testat - Versuch H - Spektralphotometer 
\vspace{2mm}
}}}
\hspace{2mm}
\makebox[0.25\textwidth]{Name:\enspace\hrulefill} \hspace{5mm}
\makebox[0.2\textwidth]{Datum:\enspace\hrulefill}
\vspace{4mm}

\begin{questions}

\question Das Lambert-Beersche Gesetz besagt, dass die Extinktion ...

\begin{choices}
	\choice linear von der Schichtdicke d abhängt.
	\choice mit zunehmender Konzentration der Lösung kleiner wird.
	\choice unabhängig von der Wellenlänge des Lichtes ist.
	\choice nicht von der Zusammensetzung der Lösung abhängt.
	\choice exponentiell mit steigender Konzentration der Lösung abnimmt.
\end{choices}

\vspace{3mm}\question Die Lichtgeschwindigkeit im Vakuum betr¨agt ungefähr:

\begin{choices}
	\choice \( 3 \cdot10^5 km/h \)
	\choice Keine der Aussagen trifft zu.
	\choice \( 3 \cdot10^5 m/h \)
	\choice \( 3 \cdot10^5 m/s \)
	\choice \( 3 \cdot10^5 km/s \)
\end{choices}

\vspace{3mm}\question Eine Farbstofflösung transmittiert 10 % des auftreffenden monochromatischen Lichts. Die Extinktion \( E=lg \frac{I_0}{I} \) dieser Lösung beträgt dann

\begin{choices}
	\choice 0,01
	\choice 1
	\choice 99
	\choice 2
	\choice 0,99
\end{choices}

\vspace{3mm}\question Unter einem Spektrum versteht man die Intensität der elektromagnetischen Welle als Funktion ihrer

\begin{choices}
	\choice Energie, Konzentration oder Wellenlänge.
	\choice Frequenz, Energie oder Konzentration.
	\choice Frequenz, Energie oder Wellenlänge.
	\choice Frequenz, Extinktion oder Wellenlänge.
	\choice Extinktion, Konzentration oder Wellenlänge.
\end{choices}

\vspace{3mm}\question Wenn \( \lambda_1 > \lambda_2 \), dann gilt für  \( \lambda_1 \)

\begin{choices}
	\choice Kleinere Frequenz, gleiche Energie wie \( \lambda_2 \).
	\choice Kleinere Frequenz, höhere Energie als \( \lambda_2 \).
	\choice Höhere Frequenz, kleinere Energie als \( \lambda_2 \).
	\choice Höhere Frequenz, gleiche Energie wie \( \lambda_2 \).
	\choice Kleinere Frequenz, kleinere Energie als \( \lambda_2 \).
\end{choices}

\vspace{3mm}\end{questions}

\end{document}
