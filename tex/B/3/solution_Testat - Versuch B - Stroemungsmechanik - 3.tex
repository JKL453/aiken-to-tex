\documentclass[11pt]{exam}

\usepackage{amsmath}
\usepackage{graphicx}
\usepackage{geometry}
\usepackage{etoolbox}
\BeforeBeginEnvironment{choices}{\par\nopagebreak\minipage{\linewidth}}
\AfterEndEnvironment{choices}{\endminipage}
\geometry{
a4paper,
total={185mm,257mm},
left=10mm,
top=25mm,
bottom=10mm
}

\begin{document}
\setlength{\voffset}{-0.5in}
\setlength{\headsep}{5pt}

\fbox{\fbox{\parbox{8cm}{\centering
\vspace{2mm}
Testat - Versuch B - Stroemungsmechanik - 3
\vspace{2mm}
}}}
\hspace{2mm}
\makebox[0.25\textwidth]{Name:\enspace\hrulefill} \hspace{5mm}
\makebox[0.2\textwidth]{Datum:\enspace\hrulefill}
\vspace{4mm}

\begin{questions}

\question Nach dem Gesetz von Hagen und Poiseuille für zylindrische Rohre

\begin{choices}
	\choice hat die Viskosität keinen Einfluss auf den Volumenstrom.
	\choice ist der Volumenstrom bei Verdopplung von Länge und Querschnittsfläche konstant.
	\choice lässt sich der Strömungswiderstand berechnen. (correct)
	\choice sind alle Strömungen turbulent.
	\choice wächst der Volumenstrom mit dem Kehrwert der Druckdifferenz.
\end{choices}

\vspace{3mm}\question Wie groß ist die Fläche eines Stempels, wenn eine Kraft von 20 N einen Druck von 100 Pa ausübt?

\begin{choices}
	\choice 0,5 qm
	\choice 0,2 qm (correct)
	\choice 200 qm
	\choice 5 qm
	\choice 2000 qm
\end{choices}

\vspace{3mm}\question Welcher Zusammenhang besteht zwischen Druck, Kraft und Fläche?

\begin{choices}
	\choice Druck ist gleich Kraft mal Fläche.
	\choice Druck ist gleich Kraft pro Fläche. (correct)
	\choice Kraft ist gleich Fläche pro Druck.
	\choice Kraft ist gleich Druck pro Fläche.
	\choice Fläche ist gleich Kraft mal Druck.
\end{choices}

\vspace{3mm}\question Die Volumenstromstärke

\begin{choices}
	\choice ist gleich der Druckdifferenz zwischen Rohranfang und Rohrende.
	\choice gibt an, welches Volumen pro Zeiteinheit durch einen Rohrquerschnitt fließt. (correct)
	\choice ist gleich der Zeit, in der ein Flüssigkeitsmolekül durch das Rohr fließt.
	\choice gibt die Geschwindigkeit der Füssigkeitsmoleküle an.
	\choice ist gleich dem Produkt aus Rohrquerschnitt und Rohrlänge.
\end{choices}

\vspace{3mm}\question Bei konstantem Blutdruck hängt die Blutmenge, die pro Minute durch ein Gefäß fließt,

\begin{choices}
	\choice besonders empfindlich von einer Änderung des Gefäßradius ab. (correct)
	\choice besonders empfindlich von einer Änderung der Länge des Blutgefäß es ab.
	\choice besonders empfindlich von einer Änderung der Kreiszahl \(\pi\) ab.
	\choice besonders empfindlich von einer Änderung der Zähflüssigkeit des Blutes ab.
	\choice besonders empfindlich von einer Änderung der Viskosität des Blutes ab.
\end{choices}

\vspace{3mm}\end{questions}

\end{document}
