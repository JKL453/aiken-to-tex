\documentclass[11pt]{exam}

\usepackage{geometry}
\geometry{
a4paper,
total={185mm,257mm},
left=10mm,
top=25mm,
bottom=10mm
}

\begin{document}
\setlength{\voffset}{-0.5in}
\setlength{\headsep}{5pt}

\fbox{\fbox{\parbox{8cm}{\centering
\vspace{2mm}
Testat - Versuch B - Stroemungsmechanik - 3
\vspace{2mm}
}}}
\hspace{2mm}
\makebox[0.25\textwidth]{Name:\enspace\hrulefill} \hspace{5mm}
\makebox[0.2\textwidth]{Datum:\enspace\hrulefill}
\vspace{4mm}

\begin{questions}

\question Nach dem Gesetz von Hagen und Poiseuille für zylindrische Rohre

\begin{choices}
	\choice lässt sich der Strömungswiderstand berechnen. (correct)
	\choice wächst der Volumenstrom mit dem Kehrwert der Druckdifferenz.
	\choice sind alle Strömungen turbulent.
	\choice ist der Volumenstrom bei Verdopplung von Länge und Querschnittsfläche konstant.
	\choice hat die Viskosität keinen Einfluss auf den Volumenstrom.
\end{choices}

\vspace{3mm}\question Auf einen Stempel mit einer Fläche von 10 qm wird eine Kraft von 5 N ausgeübt.

\begin{choices}
	\choice Der Stempel übt auf die darunterliegende Flüssigkeit einen Druck von 5 Pa aus.
	\choice Der Stempel übt auf die darunterliegende Flüssigkeit einen Druck von 25 Pa aus.
	\choice Der Stempel übt auf die darunterliegende Flüssigkeit einen Druck von 250 Pa aus.
	\choice Der Stempel übt auf die darunterliegende Flüssigkeit einen Druck von 0,5 Pa aus. (correct)
	\choice Der Stempel übt auf die darunterliegende Flüssigkeit einen Druck von 50 Pa aus.
\end{choices}

\vspace{3mm}\question Wenn der Durchmesser eines Rohres halbiert wird, dann

\begin{choices}
	\choice verdoppelt sich der Radius.
	\choice wird auch die Länge geviertelt.
	\choice wird der Durchmesser auch geviertelt.
	\choice wird der Radius halbiert. (correct)
	\choice bleibt der Durchmesser unverändert.
\end{choices}

\vspace{3mm}\question Wie verhält sich der Strömungswiderstand \(R\) eines Rohres zum Leitwert \(G\)?

\begin{choices}
	\choice Der Leitwert ist die Differenz aus Druck und Strömungswiderstand.
	\choice Der Leitwert ist gleich dem Quadrat des Stömungswiderstands.
	\choice Der Leitwert ist der Kehrwert des Strömungswiderstandes (correct)
	\choice Der Strömungswiderstand ist gleich dem Leitwert.
	\choice Zwischen beiden Grenzen besteht kein Zusammenhang.
\end{choices}

\vspace{3mm}\question Bei konstantem Blutdruck hängt die Blutmenge, die pro Minute durch ein Gefäß fließt,

\begin{choices}
	\choice besonders empfindlich von einer Änderung der Kreiszahl \(\pi\) ab.
	\choice besonders empfindlich von einer Änderung der Zähflüssigkeit des Blutes ab.
	\choice besonders empfindlich von einer Änderung der Viskosität des Blutes ab.
	\choice besonders empfindlich von einer Änderung des Gefäßradius ab. (correct)
	\choice besonders empfindlich von einer Änderung der Länge des Blutgefäß es ab.
\end{choices}

\vspace{3mm}\end{questions}

\end{document}
