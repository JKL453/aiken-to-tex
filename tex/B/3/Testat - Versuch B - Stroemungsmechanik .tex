\documentclass[11pt]{exam}

\usepackage{amsmath}
\usepackage{graphicx}
\usepackage{geometry}
\usepackage{etoolbox}
\BeforeBeginEnvironment{choices}{\par\nopagebreak\minipage{\linewidth}}
\AfterEndEnvironment{choices}{\endminipage}
\geometry{
a4paper,
total={185mm,257mm},
left=10mm,
top=25mm,
bottom=10mm
}

\begin{document}
\setlength{\voffset}{-0.5in}
\setlength{\headsep}{5pt}

\fbox{\fbox{\parbox{8cm}{\centering
\vspace{2mm}
Testat - Versuch B - Stroemungsmechanik 
\vspace{2mm}
}}}
\hspace{2mm}
\makebox[0.25\textwidth]{Name:\enspace\hrulefill} \hspace{5mm}
\makebox[0.2\textwidth]{Datum:\enspace\hrulefill}
\vspace{4mm}

\begin{questions}

\question Nach dem Gesetz von Hagen und Poiseuille für zylindrische Rohre

\begin{choices}
	\choice wächst der Volumenstrom mit dem Kehrwert der Druckdifferenz.
	\choice lässt sich der Strömungswiderstand berechnen.
	\choice sind alle Strömungen turbulent.
	\choice hat die Viskosität keinen Einfluss auf den Volumenstrom.
	\choice ist der Volumenstrom bei Verdopplung von Länge und Querschnittsfläche konstant.
\end{choices}

\vspace{3mm}\question Auf einen Stempel mit einer Fläche von 10 qm wird eine Kraft von 5 N ausgeübt.

\begin{choices}
	\choice Der Stempel übt auf die darunterliegende Flüssigkeit einen Druck von 5 Pa aus.
	\choice Der Stempel übt auf die darunterliegende Flüssigkeit einen Druck von 25 Pa aus.
	\choice Der Stempel übt auf die darunterliegende Flüssigkeit einen Druck von 0,5 Pa aus.
	\choice Der Stempel übt auf die darunterliegende Flüssigkeit einen Druck von 50 Pa aus.
	\choice Der Stempel übt auf die darunterliegende Flüssigkeit einen Druck von 250 Pa aus.
\end{choices}

\vspace{3mm}\question Welcher Zusammenhang besteht zwischen Druck, Kraft und Fläche?

\begin{choices}
	\choice Kraft ist gleich Fläche pro Druck.
	\choice Kraft ist gleich Druck pro Fläche.
	\choice Druck ist gleich Kraft pro Fläche.
	\choice Druck ist gleich Kraft mal Fläche.
	\choice Fläche ist gleich Kraft mal Druck.
\end{choices}

\vspace{3mm}\question Die Volumenstromstärke

\begin{choices}
	\choice ist gleich der Zeit, in der ein Flüssigkeitsmolekül durch das Rohr fließt.
	\choice ist gleich dem Produkt aus Rohrquerschnitt und Rohrlänge.
	\choice gibt an, welches Volumen pro Zeiteinheit durch einen Rohrquerschnitt fließt.
	\choice gibt die Geschwindigkeit der Füssigkeitsmoleküle an.
	\choice ist gleich der Druckdifferenz zwischen Rohranfang und Rohrende.
\end{choices}

\vspace{3mm}\question Bei konstantem Blutdruck hängt die Blutmenge, die pro Minute durch ein Gefäß fließt,

\begin{choices}
	\choice besonders empfindlich von einer Änderung der Kreiszahl \(\pi\) ab.
	\choice besonders empfindlich von einer Änderung der Viskosität des Blutes ab.
	\choice besonders empfindlich von einer Änderung der Zähflüssigkeit des Blutes ab.
	\choice besonders empfindlich von einer Änderung des Gefäßradius ab.
	\choice besonders empfindlich von einer Änderung der Länge des Blutgefäß es ab.
\end{choices}

\vspace{3mm}\end{questions}

\end{document}
