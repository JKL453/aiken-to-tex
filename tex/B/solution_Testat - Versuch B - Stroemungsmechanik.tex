\documentclass[11pt]{exam}

\usepackage{geometry}
\geometry{
a4paper,
total={185mm,257mm},
left=10mm,
top=25mm,
bottom=10mm
}

\begin{document}
\setlength{\voffset}{-0.5in}
\setlength{\headsep}{5pt}

\fbox{\fbox{\parbox{8cm}{\centering
\vspace{2mm}
Testat - Versuch B - Stroemungsmechanik
\vspace{2mm}
}}}
\hspace{2mm}
\makebox[0.25\textwidth]{Name:\enspace\hrulefill} \hspace{5mm}
\makebox[0.2\textwidth]{Datum:\enspace\hrulefill}
\vspace{4mm}

\begin{questions}

\question Nach dem Gesetz von Hagen und Poiseuille für zylindrische Rohre

\begin{choices}
	\choice sind alle Strömungen turbulent.
	\choice ist der Volumenstrom bei Verdopplung von Länge und Querschnittsfläche konstant.
	\choice hat die Viskosität keinen Einfluss auf den Volumenstrom.
	\choice lässt sich der Strömungswiderstand berechnen. (correct)
	\choice wächst der Volumenstrom mit dem Kehrwert der Druckdifferenz.
\end{choices}

\vspace{3mm}\question Wie groß ist die Fläche eines Stempels, wenn eine Kraft von 20 N einen Druck von 100 Pa ausübt?

\begin{choices}
	\choice 200 qm
	\choice 0,2 qm (correct)
	\choice 0,5 qm
	\choice 5 qm
	\choice 2000 qm
\end{choices}

\vspace{3mm}\question Wenn der Radius eines Rohres verdoppelt wird, dann

\begin{choices}
	\choice vervierfacht sich auch die Länge.
	\choice verdoppelt sich auch die Querschnittsfläche.
	\choice verdoppelt sich auch der Durchmesser. (correct)
	\choice vervierfacht sich auch der Durchmesser.
	\choice verdoppelt sich auch die Länge.
\end{choices}

\vspace{3mm}\question Wie verhält sich der Strömungswiderstand \(R\) eines Rohres zum Leitwert \(G\)?

\begin{choices}
	\choice Der Leitwert ist die Differenz aus Druck und Strömungswiderstand.
	\choice Zwischen beiden Grenzen besteht kein Zusammenhang.
	\choice Der Strömungswiderstand ist gleich dem Leitwert.
	\choice Der Leitwert ist gleich dem Quadrat des Stömungswiderstands.
	\choice Der Leitwert ist der Kehrwert des Strömungswiderstandes (correct)
\end{choices}

\vspace{3mm}\question Durch die Flexibilität des Aortenbogens (der wie ein Windkessel wirkt)

\begin{choices}
	\choice wird der systolische Blutdruck verringert. (correct)
	\choice werden die Spitzenwerte der Volumenstromstärke größer.
	\choice wird die Volumenstromstärke nicht beeinflußt.
	\choice wird der Blutdruck nicht beeinflußt.
	\choice wird der diastolische Blutdruck verringert.
\end{choices}

\vspace{3mm}\end{questions}

\end{document}
