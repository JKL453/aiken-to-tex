\documentclass[11pt]{exam}

\usepackage{geometry}
\geometry{
a4paper,
total={185mm,257mm},
left=10mm,
top=25mm,
bottom=10mm
}

\begin{document}
\setlength{\voffset}{-0.5in}
\setlength{\headsep}{5pt}

\fbox{\fbox{\parbox{8cm}{\centering
\vspace{2mm}
Testat - Versuch B - Stroemungsmechanik
\vspace{2mm}
}}}
\hspace{2mm}
\makebox[0.25\textwidth]{Name:\enspace\hrulefill} \hspace{5mm}
\makebox[0.2\textwidth]{Datum:\enspace\hrulefill}
\vspace{4mm}

\begin{questions}

\question Das Gesetz von Hagen und Poiseuille

\begin{choices}
	\choice gilt nur für Flüssigkeiten ohne Viskosität.
	\choice besagt, dass die Volumenstromstärke quadratisch mit der Druckdifferenz wächst.
	\choice erlaubt die Berechnung des Leitwertes eines Rohres.
	\choice sagt eine konstante Volumenstromstärke voraus, wenn sowohl Länge als auch Querschnittsfläche eines Rohrs sich halbieren.
	\choice beschreibt nur turbulente Strömungen.
\end{choices}

\vspace{3mm}\question Auf einen Stempel mit einer Fläche von 10 qm wird eine Kraft von 5 N ausgeübt.

\begin{choices}
	\choice Der Stempel übt auf die darunterliegende Flüssigkeit einen Druck von 5 Pa aus.
	\choice Der Stempel übt auf die darunterliegende Flüssigkeit einen Druck von 50 Pa aus.
	\choice Der Stempel übt auf die darunterliegende Flüssigkeit einen Druck von 0,5 Pa aus.
	\choice Der Stempel übt auf die darunterliegende Flüssigkeit einen Druck von 25 Pa aus.
	\choice Der Stempel übt auf die darunterliegende Flüssigkeit einen Druck von 250 Pa aus.
\end{choices}

\vspace{3mm}\question Wenn die Querschnittsfläche eines Rohres viermal kleiner wird, dann

\begin{choices}
	\choice wird der Radius halbiert.
	\choice bleibt der Durchmesser unverändert.
	\choice wird der Durchmesser auch geviertelt.
	\choice wird der Radius verdoppelt.
	\choice wird auch die Länge geviertelt.
\end{choices}

\vspace{3mm}\question Die Volumenstromstärke

\begin{choices}
	\choice ist gleich der Druckdifferenz zwischen Rohranfang und Rohrende.
	\choice gibt an, welches Volumen pro Zeiteinheit durch einen Rohrquerschnitt fließt.
	\choice ist gleich dem Produkt aus Rohrquerschnitt und Rohrlänge.
	\choice ist gleich der Zeit, in der ein Flüssigkeitsmolekül durch das Rohr fließt.
	\choice gibt die Geschwindigkeit der Füssigkeitsmoleküle an.
\end{choices}

\vspace{3mm}\question Bei konstantem Blutdruck hängt die Blutmenge, die pro Minute durch ein Gefäß fließt,

\begin{choices}
	\choice besonders empfindlich von einer Änderung der Kreiszahl \(\pi\) ab.
	\choice besonders empfindlich von einer Änderung des Gefäßradius ab.
	\choice besonders empfindlich von einer Änderung der Länge des Blutgefäß es ab.
	\choice besonders empfindlich von einer Änderung der Zähflüssigkeit des Blutes ab.
	\choice besonders empfindlich von einer Änderung der Viskosität des Blutes ab.
\end{choices}

\vspace{3mm}\end{questions}

\end{document}
