\documentclass[11pt]{exam}

\usepackage{amsmath}
\usepackage{graphicx}
\usepackage{geometry}
\usepackage{etoolbox}
\BeforeBeginEnvironment{choices}{\par\nopagebreak\minipage{\linewidth}}
\AfterEndEnvironment{choices}{\endminipage}
\geometry{
a4paper,
total={185mm,257mm},
left=10mm,
top=25mm,
bottom=10mm
}

\begin{document}
\setlength{\voffset}{-0.5in}
\setlength{\headsep}{5pt}

\fbox{\fbox{\parbox{8cm}{\centering
\vspace{2mm}
Testat - Versuch B - Stroemungsmechanik - 2
\vspace{2mm}
}}}
\hspace{2mm}
\makebox[0.25\textwidth]{Name:\enspace\hrulefill} \hspace{5mm}
\makebox[0.2\textwidth]{Datum:\enspace\hrulefill}
\vspace{4mm}

\begin{questions}

\question Das Gesetz von Hagen und Poiseuille

\begin{choices}
	\choice beschreibt nur turbulente Strömungen.
	\choice besagt, dass die Volumenstromstärke quadratisch mit der Druckdifferenz wächst.
	\choice gilt nur für Flüssigkeiten ohne Viskosität.
	\choice erlaubt die Berechnung des Leitwertes eines Rohres. (correct)
	\choice sagt eine konstante Volumenstromstärke voraus, wenn sowohl Länge als auch Querschnittsfläche eines Rohrs sich halbieren.
\end{choices}

\vspace{3mm}\question Auf einen Stempel mit einer Fläche von 10 qm wird eine Kraft von 5 N ausgeübt.

\begin{choices}
	\choice Der Stempel übt auf die darunterliegende Flüssigkeit einen Druck von 50 Pa aus.
	\choice Der Stempel übt auf die darunterliegende Flüssigkeit einen Druck von 0,5 Pa aus. (correct)
	\choice Der Stempel übt auf die darunterliegende Flüssigkeit einen Druck von 25 Pa aus.
	\choice Der Stempel übt auf die darunterliegende Flüssigkeit einen Druck von 5 Pa aus.
	\choice Der Stempel übt auf die darunterliegende Flüssigkeit einen Druck von 250 Pa aus.
\end{choices}

\vspace{3mm}\question Wenn der Radius eines Rohres verdoppelt wird, dann

\begin{choices}
	\choice verdoppelt sich auch die Länge.
	\choice verdoppelt sich auch der Durchmesser. (correct)
	\choice verdoppelt sich auch die Querschnittsfläche.
	\choice vervierfacht sich auch der Durchmesser.
	\choice vervierfacht sich auch die Länge.
\end{choices}

\vspace{3mm}\question Wie verhält sich der Strömungswiderstand \(R\) eines Rohres zum Leitwert \(G\)?

\begin{choices}
	\choice Der Leitwert ist gleich dem Quadrat des Stömungswiderstands.
	\choice Zwischen beiden Grenzen besteht kein Zusammenhang.
	\choice Der Strömungswiderstand ist gleich dem Leitwert.
	\choice Der Leitwert ist der Kehrwert des Strömungswiderstandes (correct)
	\choice Der Leitwert ist die Differenz aus Druck und Strömungswiderstand.
\end{choices}

\vspace{3mm}\question Durch die Flexibilität des Aortenbogens (der wie ein Windkessel wirkt)

\begin{choices}
	\choice wird die Volumenstromstärke nicht beeinflußt.
	\choice wird der Blutdruck nicht beeinflußt.
	\choice wird der diastolische Blutdruck verringert.
	\choice werden die Spitzenwerte der Volumenstromstärke größer.
	\choice wird der systolische Blutdruck verringert. (correct)
\end{choices}

\vspace{3mm}\end{questions}

\end{document}
