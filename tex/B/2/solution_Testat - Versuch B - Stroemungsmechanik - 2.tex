\documentclass[11pt]{exam}

\usepackage{geometry}
\geometry{
a4paper,
total={185mm,257mm},
left=10mm,
top=25mm,
bottom=10mm
}

\begin{document}
\setlength{\voffset}{-0.5in}
\setlength{\headsep}{5pt}

\fbox{\fbox{\parbox{8cm}{\centering
\vspace{2mm}
Testat - Versuch B - Stroemungsmechanik - 2
\vspace{2mm}
}}}
\hspace{2mm}
\makebox[0.25\textwidth]{Name:\enspace\hrulefill} \hspace{5mm}
\makebox[0.2\textwidth]{Datum:\enspace\hrulefill}
\vspace{4mm}

\begin{questions}

\question Nach dem Gesetz von Hagen und Poiseuille für zylindrische Rohre

\begin{choices}
	\choice lässt sich der Strömungswiderstand berechnen. (correct)
	\choice ist der Volumenstrom bei Verdopplung von Länge und Querschnittsfläche konstant.
	\choice wächst der Volumenstrom mit dem Kehrwert der Druckdifferenz.
	\choice sind alle Strömungen turbulent.
	\choice hat die Viskosität keinen Einfluss auf den Volumenstrom.
\end{choices}

\vspace{3mm}\question Auf einen Stempel mit einer Fläche von 3 qm wird eine Kraft von 15 N ausgeübt. Wie groß ist der Druck, den der Stempel in der darunterliegenden Flüssigkeit erzeugt?

\begin{choices}
	\choice 5 Pa (correct)
	\choice 12 Pa
	\choice 18 Pa
	\choice 45 Pa
	\choice 0,2 Pa
\end{choices}

\vspace{3mm}\question Wenn die Querschnittsfläche eines Rohres viermal kleiner wird, dann

\begin{choices}
	\choice wird auch die Länge geviertelt.
	\choice bleibt der Durchmesser unverändert.
	\choice wird der Radius verdoppelt.
	\choice wird der Radius halbiert. (correct)
	\choice wird der Durchmesser auch geviertelt.
\end{choices}

\vspace{3mm}\question Die Volumenstromstärke

\begin{choices}
	\choice ist gleich dem Produkt aus Rohrquerschnitt und Rohrlänge.
	\choice ist gleich der Druckdifferenz zwischen Rohranfang und Rohrende.
	\choice gibt die Geschwindigkeit der Füssigkeitsmoleküle an.
	\choice ist gleich der Zeit, in der ein Flüssigkeitsmolekül durch das Rohr fließt.
	\choice gibt an, welches Volumen pro Zeiteinheit durch einen Rohrquerschnitt fließt. (correct)
\end{choices}

\vspace{3mm}\question Bei konstantem Blutdruck hängt die Blutmenge, die pro Minute durch ein Gefäß fließt,

\begin{choices}
	\choice besonders empfindlich von einer Änderung des Gefäßradius ab. (correct)
	\choice besonders empfindlich von einer Änderung der Zähflüssigkeit des Blutes ab.
	\choice besonders empfindlich von einer Änderung der Länge des Blutgefäß es ab.
	\choice besonders empfindlich von einer Änderung der Viskosität des Blutes ab.
	\choice besonders empfindlich von einer Änderung der Kreiszahl \(\pi\) ab.
\end{choices}

\vspace{3mm}\end{questions}

\end{document}
