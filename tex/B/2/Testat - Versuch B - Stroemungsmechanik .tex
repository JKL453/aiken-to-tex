\documentclass[11pt]{exam}

\usepackage{amsmath}
\usepackage{graphicx}
\usepackage{geometry}
\usepackage{etoolbox}
\BeforeBeginEnvironment{choices}{\par\nopagebreak\minipage{\linewidth}}
\AfterEndEnvironment{choices}{\endminipage}
\geometry{
a4paper,
total={185mm,257mm},
left=10mm,
top=25mm,
bottom=10mm
}

\begin{document}
\setlength{\voffset}{-0.5in}
\setlength{\headsep}{5pt}

\fbox{\fbox{\parbox{8cm}{\centering
\vspace{2mm}
Testat - Versuch B - Stroemungsmechanik 
\vspace{2mm}
}}}
\hspace{2mm}
\makebox[0.25\textwidth]{Name:\enspace\hrulefill} \hspace{5mm}
\makebox[0.2\textwidth]{Datum:\enspace\hrulefill}
\vspace{4mm}

\begin{questions}

\question Das Gesetz von Hagen und Poiseuille

\begin{choices}
	\choice beschreibt nur turbulente Strömungen.
	\choice sagt eine konstante Volumenstromstärke voraus, wenn sowohl Länge als auch Querschnittsfläche eines Rohrs sich halbieren.
	\choice gilt nur für Flüssigkeiten ohne Viskosität.
	\choice erlaubt die Berechnung des Leitwertes eines Rohres.
	\choice besagt, dass die Volumenstromstärke quadratisch mit der Druckdifferenz wächst.
\end{choices}

\vspace{3mm}\question Auf einen Stempel mit einer Fläche von 10 qm wird eine Kraft von 5 N ausgeübt.

\begin{choices}
	\choice Der Stempel übt auf die darunterliegende Flüssigkeit einen Druck von 250 Pa aus.
	\choice Der Stempel übt auf die darunterliegende Flüssigkeit einen Druck von 50 Pa aus.
	\choice Der Stempel übt auf die darunterliegende Flüssigkeit einen Druck von 5 Pa aus.
	\choice Der Stempel übt auf die darunterliegende Flüssigkeit einen Druck von 25 Pa aus.
	\choice Der Stempel übt auf die darunterliegende Flüssigkeit einen Druck von 0,5 Pa aus.
\end{choices}

\vspace{3mm}\question Wenn die Querschnittsfläche eines Rohres viermal kleiner wird, dann

\begin{choices}
	\choice wird der Radius halbiert.
	\choice bleibt der Durchmesser unverändert.
	\choice wird der Radius verdoppelt.
	\choice wird auch die Länge geviertelt.
	\choice wird der Durchmesser auch geviertelt.
\end{choices}

\vspace{3mm}\question Wie verhält sich der Strömungswiderstand \(R\) eines Rohres zum Leitwert \(G\)?

\begin{choices}
	\choice Der Leitwert ist der Kehrwert des Strömungswiderstandes
	\choice Der Leitwert ist gleich dem Quadrat des Stömungswiderstands.
	\choice Der Strömungswiderstand ist gleich dem Leitwert.
	\choice Der Leitwert ist die Differenz aus Druck und Strömungswiderstand.
	\choice Zwischen beiden Grenzen besteht kein Zusammenhang.
\end{choices}

\vspace{3mm}\question Bei konstantem Blutdruck hängt die Blutmenge, die pro Minute durch ein Gefäß fließt,

\begin{choices}
	\choice besonders empfindlich von einer Änderung der Länge des Blutgefäß es ab.
	\choice besonders empfindlich von einer Änderung des Gefäßradius ab.
	\choice besonders empfindlich von einer Änderung der Zähflüssigkeit des Blutes ab.
	\choice besonders empfindlich von einer Änderung der Viskosität des Blutes ab.
	\choice besonders empfindlich von einer Änderung der Kreiszahl \(\pi\) ab.
\end{choices}

\vspace{3mm}\end{questions}

\end{document}
