\documentclass[11pt]{exam}

\usepackage{geometry}
\geometry{
a4paper,
total={185mm,257mm},
left=10mm,
top=25mm,
bottom=10mm
}

\begin{document}
\setlength{\voffset}{-0.5in}
\setlength{\headsep}{5pt}

\fbox{\fbox{\parbox{8cm}{\centering
\vspace{2mm}
Testat - Versuch B - Stroemungsmechanik - 1
\vspace{2mm}
}}}
\hspace{2mm}
\makebox[0.25\textwidth]{Name:\enspace\hrulefill} \hspace{5mm}
\makebox[0.2\textwidth]{Datum:\enspace\hrulefill}
\vspace{4mm}

\begin{questions}

\question Das Gesetz von Hagen und Poiseuille

\begin{choices}
	\choice erlaubt die Berechnung des Leitwertes eines Rohres. (correct)
	\choice sagt eine konstante Volumenstromstärke voraus, wenn sowohl Länge als auch Querschnittsfläche eines Rohrs sich halbieren.
	\choice besagt, dass die Volumenstromstärke quadratisch mit der Druckdifferenz wächst.
	\choice beschreibt nur turbulente Strömungen.
	\choice gilt nur für Flüssigkeiten ohne Viskosität.
\end{choices}

\vspace{3mm}\question Auf einen Stempel mit einer Fläche von 3 qm wird eine Kraft von 15 N ausgeübt. Wie groß ist der Druck, den der Stempel in der darunterliegenden Flüssigkeit erzeugt?

\begin{choices}
	\choice 12 Pa
	\choice 18 Pa
	\choice 0,2 Pa
	\choice 5 Pa (correct)
	\choice 45 Pa
\end{choices}

\vspace{3mm}\question Welcher Zusammenhang besteht zwischen Druck, Kraft und Fläche?

\begin{choices}
	\choice Kraft ist gleich Fläche pro Druck.
	\choice Druck ist gleich Kraft pro Fläche. (correct)
	\choice Druck ist gleich Kraft mal Fläche.
	\choice Kraft ist gleich Druck pro Fläche.
	\choice Fläche ist gleich Kraft mal Druck.
\end{choices}

\vspace{3mm}\question Die Volumenstromstärke

\begin{choices}
	\choice ist gleich der Druckdifferenz zwischen Rohranfang und Rohrende.
	\choice ist gleich dem Produkt aus Rohrquerschnitt und Rohrlänge.
	\choice ist gleich der Zeit, in der ein Flüssigkeitsmolekül durch das Rohr fließt.
	\choice gibt an, welches Volumen pro Zeiteinheit durch einen Rohrquerschnitt fließt. (correct)
	\choice gibt die Geschwindigkeit der Füssigkeitsmoleküle an.
\end{choices}

\vspace{3mm}\question Im menschlichen Gefäßsystem strömt Blut

\begin{choices}
	\choice immer laminar.
	\choice manchmal auch turbulent. (correct)
	\choice wie eine Flüssigkeit mit einer Viskosität von Null.
	\choice wie eine Flüssigkeit ohne innere Reibung.
	\choice immer turbulent.
\end{choices}

\vspace{3mm}\end{questions}

\end{document}
