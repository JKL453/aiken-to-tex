\documentclass[11pt]{exam}

\usepackage{amsmath}
\usepackage{graphicx}
\usepackage{geometry}
\usepackage{etoolbox}
\BeforeBeginEnvironment{choices}{\par\nopagebreak\minipage{\linewidth}}
\AfterEndEnvironment{choices}{\endminipage}
\geometry{
a4paper,
total={185mm,257mm},
left=10mm,
top=25mm,
bottom=10mm
}

\begin{document}
\setlength{\voffset}{-0.5in}
\setlength{\headsep}{5pt}

\fbox{\fbox{\parbox{8cm}{\centering
\vspace{2mm}
Testat - Versuch B - Stroemungsmechanik 
\vspace{2mm}
}}}
\hspace{2mm}
\makebox[0.25\textwidth]{Name:\enspace\hrulefill} \hspace{5mm}
\makebox[0.2\textwidth]{Datum:\enspace\hrulefill}
\vspace{4mm}

\begin{questions}

\question Nach dem Gesetz von Hagen und Poiseuille für zylindrische Rohre

\begin{choices}
	\choice ist der Volumenstrom bei Verdopplung von Länge und Querschnittsfläche konstant.
	\choice hat die Viskosität keinen Einfluss auf den Volumenstrom.
	\choice lässt sich der Strömungswiderstand berechnen.
	\choice sind alle Strömungen turbulent.
	\choice wächst der Volumenstrom mit dem Kehrwert der Druckdifferenz.
\end{choices}

\vspace{3mm}\question Wie groß ist die Fläche eines Stempels, wenn eine Kraft von 20 N einen Druck von 100 Pa ausübt?

\begin{choices}
	\choice 0,5 qm
	\choice 0,2 qm
	\choice 200 qm
	\choice 5 qm
	\choice 2000 qm
\end{choices}

\vspace{3mm}\question Wenn der Durchmesser eines Rohres halbiert wird, dann

\begin{choices}
	\choice wird auch die Länge geviertelt.
	\choice verdoppelt sich der Radius.
	\choice wird der Durchmesser auch geviertelt.
	\choice bleibt der Durchmesser unverändert.
	\choice wird der Radius halbiert.
\end{choices}

\vspace{3mm}\question Wie verhält sich der Strömungswiderstand \(R\) eines Rohres zum Leitwert \(G\)?

\begin{choices}
	\choice Der Leitwert ist die Differenz aus Druck und Strömungswiderstand.
	\choice Der Leitwert ist der Kehrwert des Strömungswiderstandes
	\choice Der Leitwert ist gleich dem Quadrat des Stömungswiderstands.
	\choice Zwischen beiden Grenzen besteht kein Zusammenhang.
	\choice Der Strömungswiderstand ist gleich dem Leitwert.
\end{choices}

\vspace{3mm}\question Bei konstantem Blutdruck hängt die Blutmenge, die pro Minute durch ein Gefäß fließt,

\begin{choices}
	\choice besonders empfindlich von einer Änderung der Kreiszahl \(\pi\) ab.
	\choice besonders empfindlich von einer Änderung des Gefäßradius ab.
	\choice besonders empfindlich von einer Änderung der Zähflüssigkeit des Blutes ab.
	\choice besonders empfindlich von einer Änderung der Länge des Blutgefäß es ab.
	\choice besonders empfindlich von einer Änderung der Viskosität des Blutes ab.
\end{choices}

\vspace{3mm}\end{questions}

\end{document}
