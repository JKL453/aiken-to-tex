\documentclass[11pt]{exam}

\usepackage{geometry}
\geometry{
a4paper,
total={185mm,257mm},
left=10mm,
top=25mm,
bottom=10mm
}

\begin{document}
\setlength{\voffset}{-0.5in}
\setlength{\headsep}{5pt}

\fbox{\fbox{\parbox{8cm}{\centering
\vspace{2mm}
Testat - Versuch D - Zeitabhaengiger Strom - 1
\vspace{2mm}
}}}
\hspace{2mm}
\makebox[0.25\textwidth]{Name:\enspace\hrulefill} \hspace{5mm}
\makebox[0.2\textwidth]{Datum:\enspace\hrulefill}
\vspace{4mm}

\begin{questions}

\question Welche der folgenden Aussagen ist falsch?

\begin{choices}
	\choice Wenn die Spannung an einem Kondensator erhöht wird, bleibt die Kapazität konstant.
	\choice Je größer der Widerstand in einem RC-Glied ist, desto größer ist die Zeitkonstante.
	\choice Die Kapazität eines Kondensators trägt die Einheit Farad.
	\choice Je größer die Kapazität des Kondensators in einem RC-Glied ist, desto kleiner ist die Zeitkonstante.
	\choice Eine Wechselspannung wird durch Angabe der Signalform, Frequenz, Amplitude und Phase vollständig beschrieben.
\end{choices}

\vspace{3mm}\question Ein zeitlich konstanter Strom \(\mathrm{I}\) fließt in einen Kondensator mit einer Kapazität von \(\mathrm{C=1\,mF}\) und lädt diesen auf. Nach der Zeit \(\mathrm{t=100\,s}\) beträgt die Spannung am Kondensator \(\mathrm{U=100\,V}\). Wie groß war der Strom \(\mathrm{I}\)?\(\mathrm{Q=C \cdot U}\) und \(\mathrm{Ampere=Coulomb/Sekunde}\).

\begin{choices}
	\choice \(\mathrm{0,1\,mA}\)
	\choice \(\mathrm{1\,mA}\)
	\choice \(\mathrm{10\,mA}\)
	\choice \(\mathrm{1\,A}\)
	\choice \(\mathrm{100\,mA}\)
\end{choices}

\vspace{3mm}\question Welche Periodendauer hat das auf dem Oszilloskop gezeigte Signal? Der Skalierungsfaktor für die X-Achse beträgt \(\mathrm{0,5\,ms/DIV}\), der Skalierungsfaktor für die Y-Achse beträgt \(\mathrm{1\,V/DIV}\).

\begin{choices}
	\choice \(\mathrm{0,5\,ms}\)
	\choice \(\mathrm{2\,ms}\)
	\choice \(\mathrm{1\,kHz}\)
	\choice \(\mathrm{4\,ms}\)
	\choice \(\mathrm{250\,Hz}\)
\end{choices}

\vspace{3mm}\question In welchen Einheiten werden üblicherweise die Membranzeitkonstante, die Membranlängskonstante und die Reizleitgeschwindigkeit angegeben?

\begin{choices}
	\choice Membranzeitkonstante in \(\mathrm{s}\), Membranlängskonstante in \(\mathrm{mm}\), Reizleitgeschwindigkeit in \(\mathrm{m/s}\)
	\choice Membranzeitkonstante in \(\mathrm{s^{-1}}\), Membranlängskonstante in \(\mathrm{mm^{-1}}\), Reizleitgeschwindigkeit in \(\mathrm{V/s}\)
	\choice Membranzeitkonstante und Membranlängskonstante haben keine Einheit, Reizleitgeschwindigkeit in \(\mathrm{m/s}\)
	\choice Membranzeitkonstante in \(\mathrm{s}\), Membranlängskonstante in \(\mathrm{mm}\), Reizleitgeschwindigkeit in \(\mathrm{V/s}\)
	\choice Membranzeitkonstante in \(\mathrm{s^{-1}}\), Membranlängskonstante in \(\mathrm{mm^{-1}}\), Reizleitgeschwindigkeit in \(\mathrm{m/s}\)
\end{choices}

\vspace{3mm}\question Welcher der folgenden Kurvenverläufe gibt den Zusammenhang zwischen der angelegten Spannung und der Kapazität eines Kondensators qualitativ richtig wieder?

\begin{choices}
	\choice a
	\choice b
	\choice e
	\choice c
	\choice d
\end{choices}

\vspace{3mm}\end{questions}

\end{document}
