\documentclass[11pt]{exam}

\usepackage{geometry}
\geometry{
a4paper,
total={185mm,257mm},
left=10mm,
top=25mm,
bottom=10mm
}

\begin{document}
\setlength{\voffset}{-0.5in}
\setlength{\headsep}{5pt}

\fbox{\fbox{\parbox{8cm}{\centering
\vspace{2mm}
Testat - Versuch G - Geometrische Optik
\vspace{2mm}
}}}
\hspace{2mm}
\makebox[0.25\textwidth]{Name:\enspace\hrulefill} \hspace{5mm}
\makebox[0.2\textwidth]{Datum:\enspace\hrulefill}
\vspace{4mm}

\begin{questions}

\question Wo befindet sich das Bild bei einem kurzsichtigen Auge und wie kann dies korrigiert werden?

\begin{choices}
	\choice Das Bild befindet sich vor der Netzhaut, dies wird durch eine Zerstreuungslinse korrigiert.
	\choice Das Bild befindet sich hinter der Netzhaut, dies wird durch eine Sammellinse korrigiert.
	\choice Das Bild befindet sich hinter der Netzhaut, dies wird durch eine Zerstreuungslinse korrigiert.
	\choice Das Bild ist auf der Netzhaut horizontal verschoben, dies wird durch eine Zylinderlinse korrigiert.
	\choice Das Bild befindet sich vor der Netzhaut, dies wird durch eine Sammellinse korrigiert.
\end{choices}

\vspace{3mm}\question Die Brennweite einer Linse sei f = 20 cm, Gegenstandsweite sei g = 40 cm, der Schirm befindet sich in der Entfernung b = 20 cm. Wie groß muss die Gesamtbrechkraft des Systems dieser Linse und einer Korrekturlinse sein, um eine scharfe Abbildung auf dem Schirm zu erhalten? (Der Abstand zwischen den Linsen ist zu vernachlässigen).

\begin{choices}
	\choice -5 dpt
	\choice -2,5 dpt
	\choice +2,5 dpt
	\choice -7,5 dpt
	\choice +7,5dpt
\end{choices}

\vspace{3mm}\question Eine Sammellinse hat eine Brechkraft von 3 dpt. In welcher Entfernung entsteht das Bild, wenn der Gegenstand 50 cm entfernt ist?

\begin{choices}
	\choice 75 cm
	\choice 150 cm
	\choice 50 cm
	\choice 25 cm
	\choice 100cm
\end{choices}

\vspace{3mm}\question Eine Sammellinse hat eine Brennweite von 30 cm. In welcher Entfernung entsteht das Bild, wenn der Gegenstand 90 cm entfernt ist?

\begin{choices}
	\choice 22,5 cm
	\choice 30 cm
	\choice 45cm
	\choice 75 cm
	\choice 90 cm
\end{choices}

\vspace{3mm}\question Wo befindet sich das Bild bei einem weitsichtigen Auge und wie kann dies korrigiert werden?

\begin{choices}
	\choice Das Bild befindet sich vor der Netzhaut, dies wird durch eine Zerstreuungslinse korrigiert.
	\choice Das Bild befindet sich hinter der Netzhaut, dies wird durch eine Sammellinse korrigiert.
	\choice Das Bild befindet sich vor der Netzhaut, dies wird durch eine Sammellinse korrigiert.
	\choice Das Bild befindet sich hinter der Netzhaut, dies wird durch eine Zerstreuungslinse korrigiert.
	\choice Das Bild ist auf der Netzhaut horizontal verschoben, dies wird durch eine Zylinderlinse korrigiert.
\end{choices}

\vspace{3mm}\end{questions}

\end{document}
