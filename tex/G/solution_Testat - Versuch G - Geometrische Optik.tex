\documentclass[11pt]{exam}

\usepackage{geometry}
\geometry{
a4paper,
total={185mm,257mm},
left=10mm,
top=25mm,
bottom=10mm
}

\begin{document}
\setlength{\voffset}{-0.5in}
\setlength{\headsep}{5pt}

\fbox{\fbox{\parbox{8cm}{\centering
\vspace{2mm}
Testat - Versuch G - Geometrische Optik
\vspace{2mm}
}}}
\hspace{2mm}
\makebox[0.25\textwidth]{Name:\enspace\hrulefill} \hspace{5mm}
\makebox[0.2\textwidth]{Datum:\enspace\hrulefill}
\vspace{4mm}

\begin{questions}

\question Eine Sammellinse hat eine Brechkraft von 3 dpt. In welcher Entfernung entsteht das Bild, wenn der Gegenstand 100 cm entfernt ist?

\begin{choices}
	\choice 75 cm
	\choice 33,3 cm
	\choice 100 cm
	\choice 150cm
	\choice 50 cm (correct)
\end{choices}

\vspace{3mm}\question Welche Grenzfläche im Auge trägt am meisten zur Gesamtbrechkraft des optischen Systems bei?

\begin{choices}
	\choice Linse-Glaskörper
	\choice Kammerwasser-Linse
	\choice Kammerwasser-Regenbogenhaut
	\choice Hornhaut-Kammerwasser
	\choice Luft-Hornhaut (correct)
\end{choices}

\vspace{3mm}\question Ein in Luft normalsichtiger kann unter Wasser ohne Hilfsmittel (z.B. Taucherbrille) nicht scharf sehen. Die Ursache hierfür ist:

\begin{choices}
	\choice Verstärkung der Lichtstreuung an der Cornea-Vorderseite.
	\choice Verminderung der Lichtbrechung an der Cornea-Vorderfläche. (correct)
	\choice Abnahme der Cornea-Krümmung durch den Wasser-Druck.
	\choice Trübung der Cornea durch eindringendes Wasser.
	\choice Zunahme der Cornea-Krümmung durch den Wasserdruck.
\end{choices}

\vspace{3mm}\question Wann wird ein paralleles Strahlenbündel gebrochen?

\begin{choices}
	\choice Wenn es unter senkrechtem Einfall auf eine gekrümmte Grenzfläche zweier transparenter Medien mitunterschiedlichem Brechungsindex trifft. (correct)
	\choice Wenn es unter nicht senkrechtem Einfall auf die Grenzfläche eines transparenten und eines nicht-transparenten Mediums trifft.
	\choice Wenn es unter senkrechtem Einfall auf eine gekrümmte Grenzfläche zweier transparenter Medien mit gleichem Brechungsindex trifft.
	\choice Wenn es in Luft unter senkrechtem Einfall auf die plane Seite einer Plankonkavlinse trifft.
	\choice Wenn es unter senkrechtem Einfall auf eine ebene Grenzfläche zweier transparenter Medien trifft, in denen Licht eine unterschiedliche Ausbreitungsgeschwindigkeit hat.
\end{choices}

\vspace{3mm}\question Wenn sich der Gegenstand im Brennpunkt einer Sammellinse befindet, wo entsteht das Bild hinter der Linse?

\begin{choices}
	\choice Es gibt kein Bild, der Strahl ist divergent.
	\choice Hinter der Linse zwischen Linsenoberfläche und Brennweite.
	\choice Es gibt kein Bild hinter der Linse, es entsteht ein paralleles Strahlenbündel. (correct)
	\choice  Entfernung b = f/g .
	\choice Im Bildpunkt, weiter entfernt als die Brennweite.
\end{choices}

\vspace{3mm}\end{questions}

\end{document}
