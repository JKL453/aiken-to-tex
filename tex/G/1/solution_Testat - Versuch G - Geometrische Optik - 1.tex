\documentclass[11pt]{exam}

\usepackage{geometry}
\geometry{
a4paper,
total={185mm,257mm},
left=10mm,
top=25mm,
bottom=10mm
}

\begin{document}
\setlength{\voffset}{-0.5in}
\setlength{\headsep}{5pt}

\fbox{\fbox{\parbox{8cm}{\centering
\vspace{2mm}
Testat - Versuch G - Geometrische Optik - 1
\vspace{2mm}
}}}
\hspace{2mm}
\makebox[0.25\textwidth]{Name:\enspace\hrulefill} \hspace{5mm}
\makebox[0.2\textwidth]{Datum:\enspace\hrulefill}
\vspace{4mm}

\begin{questions}

\question Was versteht man unter dem Brennpunkt?

\begin{choices}
	\choice Der Punkt in dem die optische Achse das Zentrum der Linse durchstößt.
	\choice Keine der Alternativen ist richtig.
	\choice Der Punkt in dem Strahlen aus dem Unendlichen von der Linse gesammelt werden. (correct)
	\choice Der Punkt in dem sich die Strahlen des betrachteten Gegenstandes hinter der Linse sammeln.
	\choice Das Zentrum der Lichtquelle.
\end{choices}

\vspace{3mm}\question Wenn sich der Gegenstand im Brennpunkt einer Sammellinse befindet, wo entsteht das Bild hinter der Linse?

\begin{choices}
	\choice Es gibt kein Bild hinter der Linse, es entsteht ein paralleles Strahlenbündel. (correct)
	\choice  Entfernung b = f/g .
	\choice Hinter der Linse zwischen Linsenoberfläche und Brennweite.
	\choice Im Bildpunkt, weiter entfernt als die Brennweite.
	\choice Es gibt kein Bild, der Strahl ist divergent.
\end{choices}

\vspace{3mm}\question Die Brennweite einer Linse sei f = 20 cm, Gegenstandsweite sei g = 40 cm, der Schirm befindet sich in der Entfernung b = 20 cm. Wie groß muss die Gesamtbrechkraft des Systems dieser Linse und einer Korrekturlinse sein, um eine scharfe Abbildung auf dem Schirm zu erhalten? (Der Abstand zwischen den Linsen ist zu vernachlässigen).

\begin{choices}
	\choice +2,5 dpt
	\choice -5 dpt
	\choice -2,5 dpt
	\choice -7,5 dpt
	\choice +7,5dpt (correct)
\end{choices}

\vspace{3mm}\question Wo befindet sich das Bild bei einem weitsichtigen Auge und wie kann dies korrigiert werden?

\begin{choices}
	\choice Das Bild befindet sich hinter der Netzhaut, dies wird durch eine Zerstreuungslinse korrigiert.
	\choice Das Bild befindet sich hinter der Netzhaut, dies wird durch eine Sammellinse korrigiert. (correct)
	\choice Das Bild befindet sich vor der Netzhaut, dies wird durch eine Sammellinse korrigiert.
	\choice Das Bild befindet sich vor der Netzhaut, dies wird durch eine Zerstreuungslinse korrigiert.
	\choice Das Bild ist auf der Netzhaut horizontal verschoben, dies wird durch eine Zylinderlinse korrigiert.
\end{choices}

\vspace{3mm}\question Ein in Luft normalsichtiger kann unter Wasser ohne Hilfsmittel (z.B. Taucherbrille) nicht scharf sehen. Die Ursache hierfür ist:

\begin{choices}
	\choice Verstärkung der Lichtstreuung an der Cornea-Vorderseite.
	\choice Verminderung der Lichtbrechung an der Cornea-Vorderfläche. (correct)
	\choice Trübung der Cornea durch eindringendes Wasser.
	\choice Zunahme der Cornea-Krümmung durch den Wasserdruck.
	\choice Abnahme der Cornea-Krümmung durch den Wasser-Druck.
\end{choices}

\vspace{3mm}\end{questions}

\end{document}
