\documentclass[11pt]{exam}

\usepackage{geometry}
\geometry{
a4paper,
total={185mm,257mm},
left=10mm,
top=25mm,
bottom=10mm
}

\begin{document}
\setlength{\voffset}{-0.5in}
\setlength{\headsep}{5pt}

\fbox{\fbox{\parbox{8cm}{\centering
\vspace{2mm}
Testat - Versuch G - Geometrische Optik - 3
\vspace{2mm}
}}}
\hspace{2mm}
\makebox[0.25\textwidth]{Name:\enspace\hrulefill} \hspace{5mm}
\makebox[0.2\textwidth]{Datum:\enspace\hrulefill}
\vspace{4mm}

\begin{questions}

\question In welchem Verhältnis stehen Brechkraft und Brennweite?

\begin{choices}
	\choice Sie ergeben miteinander multipliziert 1. (correct)
	\choice Wenn man die Brechkraft zur Brennweite addiert kommt man auf die Bildweite.
	\choice Wenn man die Brechkraft von der Brennweite subtrahiert ergeben sie 0.
	\choice Die Summe ergibt 1.
	\choice Die Brechkraft multipliziert mit dem Brechungsindex ergibt die Brennweite.
\end{choices}

\vspace{3mm}\question Wo befindet sich das Bild bei einem weitsichtigen Auge und wie kann dies korrigiert werden?

\begin{choices}
	\choice Das Bild befindet sich vor der Netzhaut, dies wird durch eine Sammellinse korrigiert.
	\choice Das Bild befindet sich vor der Netzhaut, dies wird durch eine Zerstreuungslinse korrigiert.
	\choice Das Bild befindet sich hinter der Netzhaut, dies wird durch eine Zerstreuungslinse korrigiert.
	\choice Das Bild befindet sich hinter der Netzhaut, dies wird durch eine Sammellinse korrigiert. (correct)
	\choice Das Bild ist auf der Netzhaut horizontal verschoben, dies wird durch eine Zylinderlinse korrigiert.
\end{choices}

\vspace{3mm}\question Die Brennweite einer Linse sei f = 20 cm, Gegenstandsweite sei g = 40 cm, der Schirm befindet sich in der Entfernung b = 20 cm. Wie groß muss die Gesamtbrechkraft des Systems dieser Linse und einer Korrekturlinse sein, um eine scharfe Abbildung auf dem Schirm zu erhalten? (Der Abstand zwischen den Linsen ist zu vernachlässigen).

\begin{choices}
	\choice -5 dpt
	\choice +2,5 dpt
	\choice -7,5 dpt
	\choice -2,5 dpt
	\choice +7,5dpt (correct)
\end{choices}

\vspace{3mm}\question In welchem Verhältnis stehen Brechkraft und Brennweite?

\begin{choices}
	\choice Die Summe ergibt 1.
	\choice Wenn man die Brechkraft zur Brennweite addiert kommt man auf die Bildweite.
	\choice Wenn man die Brechkraft von der Brennweite subtrahiert ergeben sie 0.
	\choice Sie ergeben miteinander multipliziert 1. (correct)
	\choice Die Brechkraft multipliziert mit dem Brechungsindex ergibt die Brennweite.
\end{choices}

\vspace{3mm}\question Wenn der Gegenstand sich weiter entfernt als der Brennpunkt einer Sammellinse befindet, wo entsteht das Bild hinter der Linse?

\begin{choices}
	\choice Wenn sich der Gegenstand in einer Entfernung g vor der Linse befindet, befindet sich das Bild in derEntfernung b = f/g .
	\choice Hinter der Linse zwischen Linsenoberfläche und Brennweite.
	\choice Im Bildpunkt, weiter entfernt als die Brennweite. (correct)
	\choice Im Unendlichen, es bildet sich ein paralleles Strahlenbündel.
	\choice Es gibt kein Bild, der Strahl ist divergent.
\end{choices}

\vspace{3mm}\end{questions}

\end{document}
