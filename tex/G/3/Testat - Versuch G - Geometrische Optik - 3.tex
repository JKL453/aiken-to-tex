\documentclass[11pt]{exam}

\usepackage{amsmath}
\usepackage{graphicx}
\usepackage{geometry}
\usepackage{etoolbox}
\BeforeBeginEnvironment{choices}{\par\nopagebreak\minipage{\linewidth}}
\AfterEndEnvironment{choices}{\endminipage}
\geometry{
a4paper,
total={185mm,257mm},
left=10mm,
top=25mm,
bottom=10mm
}

\begin{document}
\setlength{\voffset}{-0.5in}
\setlength{\headsep}{5pt}

\fbox{\fbox{\parbox{8cm}{\centering
\vspace{2mm}
Testat - Versuch G - Geometrische Optik - 3
\vspace{2mm}
}}}
\hspace{2mm}
\makebox[0.25\textwidth]{Name:\enspace\hrulefill} \hspace{5mm}
\makebox[0.2\textwidth]{Datum:\enspace\hrulefill}
\vspace{4mm}

\begin{questions}

\question Wenn sich der Gegenstand im Brennpunkt einer Sammellinse befindet, wo entsteht das Bild hinter der Linse?

\begin{choices}
	\choice Im Bildpunkt, weiter entfernt als die Brennweite.
	\choice Hinter der Linse zwischen Linsenoberfläche und Brennweite.
	\choice  Entfernung b = f/g .
	\choice Es gibt kein Bild hinter der Linse, es entsteht ein paralleles Strahlenbündel.
	\choice Es gibt kein Bild, der Strahl ist divergent.
\end{choices}

\vspace{3mm}\question Ein in Luft normalsichtiger kann unter Wasser ohne Hilfsmittel (z.B. Taucherbrille) nicht scharf sehen. Die Ursache hierfür ist:

\begin{choices}
	\choice Verstärkung der Lichtstreuung an der Cornea-Vorderseite.
	\choice Verminderung der Lichtbrechung an der Cornea-Vorderfläche.
	\choice Trübung der Cornea durch eindringendes Wasser.
	\choice Abnahme der Cornea-Krümmung durch den Wasser-Druck.
	\choice Zunahme der Cornea-Krümmung durch den Wasserdruck.
\end{choices}

\vspace{3mm}\question Welche Aussage trifft zu:Bei dem Übergang eines schräg einfallenden Lichtstrahls vom optisch dünneren (Medium 1) zum optisch dichteren Medium 2 wird (n1 < n2):

\begin{choices}
	\choice Der Lichtstrahl nicht abgelenkt
	\choice Stets Totalreflektion erreicht
	\choice Die Lichtgeschwindigkeit erhöht
	\choice Das Licht zum Einfallslot hin gebrochen
	\choice Das Licht zum Einfallslot weg gebrochen
\end{choices}

\vspace{3mm}\question Wie lautet die Einheit der Brechkraft?

\begin{choices}
	\choice einheitenlos
	\choice N/m
	\choice m
	\choice s
	\choice 1/m
\end{choices}

\vspace{3mm}\question Eine Sammellinse von 20 cm Brennweite soll mit einer Linse kombiniert werden, um eine Gesamtbrennweite von 40 cm zu erhalten. (Beide Linsen seien ’dünn’ und einander so nah, dass der Abstand vernachlässigbar klein ist.)Welche Linse ist erforderlich?

\begin{choices}
	\choice Sammellinse mit 60 cm Brennweite
	\choice Zerstreuungslinse mit -40 cm Brennweite
	\choice Sammellinse mit 20 cm Brennweite
	\choice  Sammellinse mit 40 cm Brennweite
	\choice Zerstreuungslinse mit -20 cm Brennweite
\end{choices}

\vspace{3mm}\end{questions}

\end{document}
