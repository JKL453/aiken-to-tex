\documentclass[11pt]{exam}

\usepackage{amsmath}
\usepackage{graphicx}
\usepackage{geometry}
\usepackage{etoolbox}
\BeforeBeginEnvironment{choices}{\par\nopagebreak\minipage{\linewidth}}
\AfterEndEnvironment{choices}{\endminipage}
\geometry{
a4paper,
total={185mm,257mm},
left=10mm,
top=25mm,
bottom=10mm
}

\begin{document}
\setlength{\voffset}{-0.5in}
\setlength{\headsep}{5pt}

\fbox{\fbox{\parbox{8cm}{\centering
\vspace{2mm}
Testat - Versuch G - Geometrische Optik 
\vspace{2mm}
}}}
\hspace{2mm}
\makebox[0.25\textwidth]{Name:\enspace\hrulefill} \hspace{5mm}
\makebox[0.2\textwidth]{Datum:\enspace\hrulefill}
\vspace{4mm}

\begin{questions}

\question Eine Sammellinse hat eine Brechkraft von 3 dpt. In welcher Entfernung entsteht das Bild, wenn der Gegenstand 100 cm entfernt ist?

\begin{choices}
	\choice 100 cm
	\choice 75 cm
	\choice 150cm
	\choice 33,3 cm
	\choice 50 cm
\end{choices}

\vspace{3mm}\question Wo befindet sich das Bild bei einem kurzsichtigen Auge und wie kann dies korrigiert werden?

\begin{choices}
	\choice Das Bild befindet sich hinter der Netzhaut, dies wird durch eine Sammellinse korrigiert.
	\choice Das Bild befindet sich vor der Netzhaut, dies wird durch eine Zerstreuungslinse korrigiert.
	\choice Das Bild ist auf der Netzhaut horizontal verschoben, dies wird durch eine Zylinderlinse korrigiert.
	\choice Das Bild befindet sich vor der Netzhaut, dies wird durch eine Sammellinse korrigiert.
	\choice Das Bild befindet sich hinter der Netzhaut, dies wird durch eine Zerstreuungslinse korrigiert.
\end{choices}

\vspace{3mm}\question Wann wird ein paralleles Strahlenbündel gebrochen?

\begin{choices}
	\choice Wenn es unter senkrechtem Einfall auf eine gekrümmte Grenzfläche zweier transparenter Medien mitunterschiedlichem Brechungsindex trifft.
	\choice Wenn es in Luft unter senkrechtem Einfall auf die plane Seite einer Plankonkavlinse trifft.
	\choice Wenn es unter senkrechtem Einfall auf eine gekrümmte Grenzfläche zweier transparenter Medien mit gleichem Brechungsindex trifft.
	\choice Wenn es unter senkrechtem Einfall auf eine ebene Grenzfläche zweier transparenter Medien trifft, in denen Licht eine unterschiedliche Ausbreitungsgeschwindigkeit hat.
	\choice Wenn es unter nicht senkrechtem Einfall auf die Grenzfläche eines transparenten und eines nicht-transparenten Mediums trifft.
\end{choices}

\vspace{3mm}\question Wie lautet die Einheit der Brechkraft?

\begin{choices}
	\choice 1/m
	\choice s
	\choice N/m
	\choice m
	\choice einheitenlos
\end{choices}

\vspace{3mm}\question Die Brennweite einer Linse sei f = 20 cm, Gegenstandsweite sei g = 40 cm, der Schirm befindet sich in der Entfernung b = 20 cm. Wie groß muss die Brechkraft einer Korrekturlinse sein, um eine scharfe Abbildung auf dem Schirm zu erhalten? (Der Abstand zwischen den Linsen ist zu vernachlässigen)

\begin{choices}
	\choice −5dpt
	\choice +5dpt
	\choice −2,5dpt
	\choice +2,5 dpt
	\choice +10/2m^−1
\end{choices}

\vspace{3mm}\end{questions}

\end{document}
