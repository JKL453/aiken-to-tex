\documentclass[11pt]{exam}

\usepackage{geometry}
\geometry{
a4paper,
total={185mm,257mm},
left=10mm,
top=25mm,
bottom=10mm
}

\begin{document}
\setlength{\voffset}{-0.5in}
\setlength{\headsep}{5pt}

\fbox{\fbox{\parbox{8cm}{\centering
\vspace{2mm}
Testat - Versuch G - Geometrische Optik - 2
\vspace{2mm}
}}}
\hspace{2mm}
\makebox[0.25\textwidth]{Name:\enspace\hrulefill} \hspace{5mm}
\makebox[0.2\textwidth]{Datum:\enspace\hrulefill}
\vspace{4mm}

\begin{questions}

\question Wenn sich der Gegenstand im Brennpunkt einer Sammellinse befindet, wo entsteht das Bild hinter der Linse?

\begin{choices}
	\choice Im Bildpunkt, weiter entfernt als die Brennweite.
	\choice Hinter der Linse zwischen Linsenoberfläche und Brennweite.
	\choice Es gibt kein Bild hinter der Linse, es entsteht ein paralleles Strahlenbündel.
	\choice Es gibt kein Bild, der Strahl ist divergent.
	\choice  Entfernung b = f/g .
\end{choices}

\vspace{3mm}\question Eine Sammellinse hat eine Brennweite von 30 cm. In welcher Entfernung entsteht das Bild, wenn der Gegenstand 45 cm entfernt ist?

\begin{choices}
	\choice 45 cm
	\choice 75 cm
	\choice 30 cm
	\choice 60 cm
	\choice 90cm
\end{choices}

\vspace{3mm}\question Eine Sammellinse hat eine Brennweite von 30 cm. In welcher Entfernung entsteht das Bild, wenn der Gegenstand 90 cm entfernt ist?

\begin{choices}
	\choice 30 cm
	\choice 90 cm
	\choice 45cm
	\choice 22,5 cm
	\choice 75 cm
\end{choices}

\vspace{3mm}\question In welchem Verhältnis stehen Brechkraft und Brennweite?

\begin{choices}
	\choice Sie ergeben miteinander multipliziert 1.
	\choice Die Brechkraft multipliziert mit dem Brechungsindex ergibt die Brennweite.
	\choice Die Summe ergibt 1.
	\choice Wenn man die Brechkraft zur Brennweite addiert kommt man auf die Bildweite.
	\choice Wenn man die Brechkraft von der Brennweite subtrahiert ergeben sie 0.
\end{choices}

\vspace{3mm}\question Eine Sammellinse von 20 cm Brennweite soll mit einer Linse kombiniert werden, um eine Gesamtbrennweite von 40 cm zu erhalten. (Beide Linsen seien ’dünn’ und einander so nah, dass der Abstand vernachlässigbar klein ist.)Welche Linse ist erforderlich?

\begin{choices}
	\choice Sammellinse mit 60 cm Brennweite
	\choice  Sammellinse mit 40 cm Brennweite
	\choice Sammellinse mit 20 cm Brennweite
	\choice Zerstreuungslinse mit -20 cm Brennweite
	\choice Zerstreuungslinse mit -40 cm Brennweite
\end{choices}

\vspace{3mm}\end{questions}

\end{document}
