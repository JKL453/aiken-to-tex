\documentclass[11pt]{exam}

\usepackage{geometry}
\geometry{
a4paper,
total={185mm,257mm},
left=10mm,
top=25mm,
bottom=10mm
}

\begin{document}
\setlength{\voffset}{-0.5in}
\setlength{\headsep}{5pt}

\fbox{\fbox{\parbox{8cm}{\centering
\vspace{2mm}
Testat - Versuch A - Waermelehre
\vspace{2mm}
}}}
\hspace{2mm}
\makebox[0.25\textwidth]{Name:\enspace\hrulefill} \hspace{5mm}
\makebox[0.2\textwidth]{Datum:\enspace\hrulefill}
\vspace{4mm}

\begin{questions}

\question Welche der folgenden Einheiten ist die SI-Einheit für die Wärmeenergie?

\begin{choices}
	\choice Grad Celsius
	\choice Watt
	\choice Newton
	\choice Kelvin
	\choice Joule (correct)
\end{choices}

\vspace{3mm}\question Wenn warme und kalte Flüssigkeiten in gleichen Mengen gemischt werden, dann hat das Gemisch

\begin{choices}
	\choice eine Temperatur, die zwischen beiden Temperaturen liegt. (correct)
	\choice die Temperatur der warmen Flüssigkeit.
	\choice eine Temperatur, die der Summe der Temperaturen der warmen und der kalten Flüssigkeit entspricht.
	\choice die Temperatur der kalten Flüssigkeit.
	\choice eine Temperatur, die der Differenz der Temperaturen der warmen und der kalten Flüssigkeit entspricht.
\end{choices}

\vspace{3mm}\question Die Temperatur eines Körpers

\begin{choices}
	\choice wächst quadratisch mit dem Volumen des Körpers.
	\choice wächst quadratisch mit der Masse des Körpers.
	\choice wächst linear mit der Masse des Körpers.
	\choice hängt mit der Wärmeenergie des Körpers zusammen. (correct)
	\choice ist umgekehrt proportional zur Wärmeenergie des Körpers.
\end{choices}

\vspace{3mm}\question Wasser hat eine spezifische Wärmekapazität von (sehr grob angenähert) 4 kJ kg\(^{-1}\) K\(^{-1}\) und eine spezifische Verdampfungswärme von etwa 2 MJ kg\(^{-1}\).Zwei Kilogramm Wasser wird bei einer Temperatur von 99\(^\circ\)C eine Wärmemenge von 2 MJ zugeführt. Als Ergebnis erhält man

\begin{choices}
	\choice etwa 1 kg heißes Wasser und etwa 1 kg Wasserdampf. (correct)
	\choice etwa 2 kg Wasserdampf.
	\choice etwa 2 kg heißes Wasser und etwa 1 kg Wasserdampf.
	\choice etwa 2 kg heißes Wasser und etwa 2 kg Wasserdampf.
	\choice etwa 2 kg heißes Wasser.
\end{choices}

\vspace{3mm}\question Welcher Zusammenhang besteht zwischen Wärme, Wärmekapazität und Temperatur? (\(m\): Masse)

\begin{choices}
	\choice \(\Delta Q = m \, (\Delta T)^2\)
	\choice \(\Delta Q = m c \Delta T\) (correct)
	\choice \(Q = m c^2\)
	\choice \(\Delta Q = m \Delta T\)
	\choice \(Q = m c\)
\end{choices}

\vspace{3mm}\end{questions}

\end{document}
