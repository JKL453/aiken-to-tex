\documentclass[11pt]{exam}

\usepackage{geometry}
\geometry{
a4paper,
total={185mm,257mm},
left=10mm,
top=25mm,
bottom=10mm
}

\begin{document}
\setlength{\voffset}{-0.5in}
\setlength{\headsep}{5pt}

\fbox{\fbox{\parbox{8cm}{\centering
\vspace{2mm}
Testat - Versuch A - Waermelehre - 2
\vspace{2mm}
}}}
\hspace{2mm}
\makebox[0.25\textwidth]{Name:\enspace\hrulefill} \hspace{5mm}
\makebox[0.2\textwidth]{Datum:\enspace\hrulefill}
\vspace{4mm}

\begin{questions}

\question Eine spezifische Verdampfungswärme von 2 J g\(^{-1}\) bedeutet, dass

\begin{choices}
	\choice eine Wärmezufuhr von 5 J zu einer Verdampfung von 5 g Flüssigkeit führt.
	\choice eine Wärmezufuhr von 2 J zu einer Verdampfung von 2 g Flüssigkeit führt.
	\choice eine Wärmezufuhr von 1 J zu einer Verdampfung von 0,5 g Flüssigkeit führt. (correct)
	\choice eine Wärmezufuhr von 1 J zu einer Verdampfung von 1 g Flüssigkeit führt.
	\choice bei einer Temperaturerhöhung um 10 K genau 5 g Flüssigkeit verdampfen.
\end{choices}

\vspace{3mm}\question Wenn warme und kalte Flüssigkeiten in gleichen Mengen gemischt werden, dann hat das Gemisch

\begin{choices}
	\choice die Wärmeenergie der kalten Flüssigkeit.
	\choice eine Wärmeenergie kleiner als die Wärmeenergie der kalten Flüssigkeit.
	\choice eine Wärmeenergie, die gleich der Summe der Wärmeenergien der beiden Flüssigkeiten ist. (correct)
	\choice eine Wärmeenergie kleiner als die Wärmeenergie der warmen Flüssigkeit.
	\choice die Wärmeenergie der warmen Flüssigkeit.
\end{choices}

\vspace{3mm}\question Welche der folgenden Aussagen ist FALSCH?

\begin{choices}
	\choice Für Temperaturdifferenzen erhält man mit der Kelvin- und der Celsiusskala die gleichen Werte.
	\choice Eine Temperaturdifferenz von 1 K entspricht einer Temperaturdifferenz von 1\(^\circ\)C.
	\choice Eine Temperatur von 0 K entspricht einer Temperatur von etwa -273,15\(^\circ\)C.
	\choice Physikalische Ausdrücke, die die absolute Temperatur enthalten, erfordern Angaben in der Kelvinskala.
	\choice Für die spezifische Wärmekapazität eines Körpers muß man unterschiedliche Werte verwenden, je nachdem, ob man mit Kelvin oder mit Grad Celsius rechnet. (correct)
\end{choices}

\vspace{3mm}\question Durch eine Wärmezufuhr von 4 J werden 2 g einer Flüssigkeit verdampft.Das heißt,

\begin{choices}
	\choice Die Wärmekapazität der Flüssigkeit beträgt 1 J g\(^{-1}\) K\(^{-1}\).
	\choice Die spezifische Verdampfungswärme beträgt 1 J g\(^{-1}\).
	\choice Die spezifische Verdampfungswärme beträgt 2 J g\(^{-1}\). (correct)
	\choice Die Wärmekapazität der Flüssigkeit beträgt 2 J g\(^{-1}\) K\(^{-1}\).
	\choice Die spezifische Verdampfungswärme beträgt 0,5 J g\(^{-1}\).
\end{choices}

\vspace{3mm}\question Welcher Zusammenhang besteht zwischen Wärme, Wärmekapazität und Temperatur? (\(m\): Masse)

\begin{choices}
	\choice \(Q = m c^2\)
	\choice \(\Delta Q = m \Delta T\)
	\choice \(Q = m c\)
	\choice \(\Delta Q = m c \Delta T\) (correct)
	\choice \(\Delta Q = m \, (\Delta T)^2\)
\end{choices}

\vspace{3mm}\end{questions}

\end{document}
