\documentclass[11pt]{exam}

\usepackage{amsmath}
\usepackage{graphicx}
\usepackage{geometry}
\usepackage{etoolbox}
\BeforeBeginEnvironment{choices}{\par\nopagebreak\minipage{\linewidth}}
\AfterEndEnvironment{choices}{\endminipage}
\geometry{
a4paper,
total={185mm,257mm},
left=10mm,
top=25mm,
bottom=10mm
}

\begin{document}
\setlength{\voffset}{-0.5in}
\setlength{\headsep}{5pt}

\fbox{\fbox{\parbox{8cm}{\centering
\vspace{2mm}
Testat - Versuch A - Waermelehre 
\vspace{2mm}
}}}
\hspace{2mm}
\makebox[0.25\textwidth]{Name:\enspace\hrulefill} \hspace{5mm}
\makebox[0.2\textwidth]{Datum:\enspace\hrulefill}
\vspace{4mm}

\begin{questions}

\question Welche der folgenden Aussagen ist FALSCH?Um die Körpertemperatur konstant zu halten,

\begin{choices}
	\choice kann der Mensch Wasser auf der Haut verdunsten lassen.
	\choice kann der Mensch warme Luft aus- und kalte Luft einatmen.
	\choice kann der Mensch Wärmeenergie in mechanische Arbeit umwandeln.
	\choice kann der Mensch Wärme durch direkten Kontakt mit kälteren Körpern abgeben.
	\choice kann der Mensch Wärme durch Infrarotstrahlung abgeben.
\end{choices}

\vspace{3mm}\question Wenn warme und kalte Flüssigkeiten in gleichen Mengen gemischt werden, dann hat das Gemisch

\begin{choices}
	\choice eine Temperatur, die zwischen beiden Temperaturen liegt.
	\choice die Temperatur der warmen Flüssigkeit.
	\choice eine Temperatur, die der Summe der Temperaturen der warmen und der kalten Flüssigkeit entspricht.
	\choice die Temperatur der kalten Flüssigkeit.
	\choice eine Temperatur, die der Differenz der Temperaturen der warmen und der kalten Flüssigkeit entspricht.
\end{choices}

\vspace{3mm}\question Wenn zwei gleich große Mengen einer Flüssigkeit mit gleicher Temperatur gemischt werden, dann

\begin{choices}
	\choice hat das Gemisch die Summe der ursprünglichen Wärmeenergien.
	\choice hat das Gemisch die gleiche Wärmeenergie wie jede der Ausgangsmengen.
	\choice hat das Gemisch eine Wärmeenergie kleiner als die Wärmeenergie von einer der Ausgangsmengen.
	\choice hat das Gemisch die vierfache Wärmeenergie wie eine der Ausgangsmengen.
	\choice hat das Gemisch die halbe Wärmeenergie wie eine der Ausgangsmengen.
\end{choices}

\vspace{3mm}\question Wasser hat eine spezifische Wärmekapazität von (sehr grob angenähert) 4 kJ kg\(^{-1}\) K\(^{-1}\) und eine spezifische Verdampfungswärme von etwa 2 MJ kg\(^{-1}\).Einem Kilogramm Wasser wird bei einer Temperatur von 99\(^\circ\)C eine Wärmemenge von 2 MJ zugeführt. Als Ergebnis erhält man

\begin{choices}
	\choice etwa 0,5 kg heißes Wasser und 0,5 kg Wasserdampf.
	\choice etwa 1 kg heißes Wasser und 1 kg Wasserdampf.
	\choice etwa 2 kg heißes Wasser.
	\choice etwa 1 kg Wasserdampf.
	\choice etwa 1 kg heißes Wasser.
\end{choices}

\vspace{3mm}\question Welcher Zusammenhang besteht zwischen Wärme, Wärmekapazität und Temperatur? (\(m\): Masse)

\begin{choices}
	\choice \(Q = m c\)
	\choice \(\Delta Q = m c \Delta T\)
	\choice \(Q = m c^2\)
	\choice \(\Delta Q = m \Delta T\)
	\choice \(\Delta Q = m \, (\Delta T)^2\)
\end{choices}

\vspace{3mm}\end{questions}

\end{document}
