\documentclass[11pt]{exam}

\usepackage{amsmath}
\usepackage{graphicx}
\usepackage{geometry}
\usepackage{etoolbox}
\BeforeBeginEnvironment{choices}{\par\nopagebreak\minipage{\linewidth}}
\AfterEndEnvironment{choices}{\endminipage}
\geometry{
a4paper,
total={185mm,257mm},
left=10mm,
top=25mm,
bottom=10mm
}

\begin{document}
\setlength{\voffset}{-0.5in}
\setlength{\headsep}{5pt}

\fbox{\fbox{\parbox{8cm}{\centering
\vspace{2mm}
Testat - Versuch A - Waermelehre - 3
\vspace{2mm}
}}}
\hspace{2mm}
\makebox[0.25\textwidth]{Name:\enspace\hrulefill} \hspace{5mm}
\makebox[0.2\textwidth]{Datum:\enspace\hrulefill}
\vspace{4mm}

\begin{questions}

\question Wasser hat eine spezifische Wärmekapazität von (sehr grob angenähert) 4 kJ kg\(^{-1}\) K\(^{-1}\) und eine spezifische Verdampfungswärme von etwa 2 MJ kg\(^{-1}\). Einem Kilogramm Wasser wird bei einer Temperatur von 99\(^\circ\)C eine Wärmemenge von 2 MJ zugeführt. Als Ergebnis erhält man

\begin{choices}
	\choice etwa 1 kg Wasserdampf.
	\choice etwa 1 kg heißes Wasser und 1 kg Wasserdampf.
	\choice etwa 1 kg heißes Wasser.
	\choice etwa 0,5 kg heißes Wasser und 0,5 kg Wasserdampf.
	\choice etwa 2 kg heißes Wasser.
\end{choices}

\vspace{3mm}\question Welche der folgenden Einheiten ist die SI-Einheit für die ABSOLUTE Temperatur?

\begin{choices}
	\choice Joule
	\choice Grad Celsius
	\choice Watt
	\choice Kilogramm
	\choice Kelvin
\end{choices}

\vspace{3mm}\question Wenn zwei in jeder Hinsicht identische Flüssigkeitsmengen gemischt werden, dann

\begin{choices}
	\choice ist die Temperatur des Gemisches halb so groß wie die Ausgangstemperatur.
	\choice ist die Temperatur des Gemisches größer als die Ausgangstemperatur.
	\choice ist die Temperatur des Gemisches doppelt so groß wie die Ausgangstemperatur.
	\choice ist die Temperatur des Gemisches kleiner als die Ausgangstemperatur.
	\choice ist die Temperatur des Gemisches genauso groß wie die Ausgangstemperatur.
\end{choices}

\vspace{3mm}\question Wasser hat eine spezifische Wärmekapazität von (sehr grob angenähert) 4 kJ kg\(^{-1}\) K\(^{-1}\) und eine spezifische Verdampfungswärme von etwa 2 MJ kg\(^{-1}\).Zwei Kilogramm Wasser wird bei einer Temperatur von 99\(^\circ\)C eine Wärmemenge von 2 MJ zugeführt. Als Ergebnis erhält man

\begin{choices}
	\choice etwa 1 kg heißes Wasser und etwa 1 kg Wasserdampf.
	\choice etwa 2 kg heißes Wasser und etwa 1 kg Wasserdampf.
	\choice etwa 2 kg heißes Wasser und etwa 2 kg Wasserdampf.
	\choice etwa 2 kg heißes Wasser.
	\choice etwa 2 kg Wasserdampf.
\end{choices}

\vspace{3mm}\question Welcher Zusammenhang besteht zwischen Wärme, Wärmekapazität und Temperatur? (\(m\): Masse)

\begin{choices}
	\choice \(Q = m c^2\)
	\choice \(Q = m c\)
	\choice \(\Delta Q = m c \Delta T\)
	\choice \(\Delta Q = m \Delta T\)
	\choice \(\Delta Q = m \, (\Delta T)^2\)
\end{choices}

\vspace{3mm}\end{questions}

\end{document}
