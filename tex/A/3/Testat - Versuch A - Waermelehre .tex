\documentclass[11pt]{exam}

\usepackage{amsmath}
\usepackage{graphicx}
\usepackage{geometry}
\usepackage{etoolbox}
\BeforeBeginEnvironment{choices}{\par\nopagebreak\minipage{\linewidth}}
\AfterEndEnvironment{choices}{\endminipage}
\geometry{
a4paper,
total={185mm,257mm},
left=10mm,
top=25mm,
bottom=10mm
}

\begin{document}
\setlength{\voffset}{-0.5in}
\setlength{\headsep}{5pt}

\fbox{\fbox{\parbox{8cm}{\centering
\vspace{2mm}
Testat - Versuch A - Waermelehre 
\vspace{2mm}
}}}
\hspace{2mm}
\makebox[0.25\textwidth]{Name:\enspace\hrulefill} \hspace{5mm}
\makebox[0.2\textwidth]{Datum:\enspace\hrulefill}
\vspace{4mm}

\begin{questions}

\question Eine spezifische Verdampfungswärme von 2 J g\(^{-1}\) bedeutet, dass

\begin{choices}
	\choice eine Wärmezufuhr von 2 J zu einer Verdampfung von 2 g Flüssigkeit führt.
	\choice bei einer Temperaturerhöhung um 10 K genau 5 g Flüssigkeit verdampfen.
	\choice eine Wärmezufuhr von 1 J zu einer Verdampfung von 1 g Flüssigkeit führt.
	\choice eine Wärmezufuhr von 5 J zu einer Verdampfung von 5 g Flüssigkeit führt.
	\choice eine Wärmezufuhr von 1 J zu einer Verdampfung von 0,5 g Flüssigkeit führt.
\end{choices}

\vspace{3mm}\question Wenn warme und kalte Flüssigkeiten in gleichen Mengen gemischt werden, dann hat das Gemisch

\begin{choices}
	\choice die Wärmeenergie der kalten Flüssigkeit.
	\choice eine Wärmeenergie kleiner als die Wärmeenergie der kalten Flüssigkeit.
	\choice die Wärmeenergie der warmen Flüssigkeit.
	\choice eine Wärmeenergie, die gleich der Summe der Wärmeenergien der beiden Flüssigkeiten ist.
	\choice eine Wärmeenergie kleiner als die Wärmeenergie der warmen Flüssigkeit.
\end{choices}

\vspace{3mm}\question Wenn warme und kalte Flüssigkeiten in gleichen Mengen gemischt werden, dann hat das Gemisch

\begin{choices}
	\choice eine Temperatur, die der Differenz der Temperaturen der warmen und der kalten Flüssigkeit entspricht.
	\choice die Temperatur der warmen Flüssigkeit.
	\choice eine Temperatur, die zwischen beiden Temperaturen liegt.
	\choice eine Temperatur, die der Summe der Temperaturen der warmen und der kalten Flüssigkeit entspricht.
	\choice die Temperatur der kalten Flüssigkeit.
\end{choices}

\vspace{3mm}\question Wasser hat eine spezifische Wärmekapazität von (sehr grob angenähert) 4 kJ kg\(^{-1}\) K\(^{-1}\) und eine spezifische Verdampfungswärme von etwa 2 MJ kg\(^{-1}\).Zwei Kilogramm Wasser wird bei einer Temperatur von 99\(^\circ\)C eine Wärmemenge von 2 MJ zugeführt. Als Ergebnis erhält man

\begin{choices}
	\choice etwa 2 kg heißes Wasser und etwa 1 kg Wasserdampf.
	\choice etwa 2 kg heißes Wasser und etwa 2 kg Wasserdampf.
	\choice etwa 2 kg heißes Wasser.
	\choice etwa 2 kg Wasserdampf.
	\choice etwa 1 kg heißes Wasser und etwa 1 kg Wasserdampf.
\end{choices}

\vspace{3mm}\question Welcher Zusammenhang besteht zwischen Wärme, Wärmekapazität und Temperatur? (\(m\): Masse)

\begin{choices}
	\choice \(Q = m c\)
	\choice \(Q = m c^2\)
	\choice \(\Delta Q = m \, (\Delta T)^2\)
	\choice \(\Delta Q = m c \Delta T\)
	\choice \(\Delta Q = m \Delta T\)
\end{choices}

\vspace{3mm}\end{questions}

\end{document}
