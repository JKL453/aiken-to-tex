\documentclass[11pt]{exam}

\usepackage{geometry}
\geometry{
a4paper,
total={185mm,257mm},
left=10mm,
top=25mm,
bottom=10mm
}

\begin{document}
\setlength{\voffset}{-0.5in}
\setlength{\headsep}{5pt}

\fbox{\fbox{\parbox{8cm}{\centering
\vspace{2mm}
Testat - Versuch A - Waermelehre - 1
\vspace{2mm}
}}}
\hspace{2mm}
\makebox[0.25\textwidth]{Name:\enspace\hrulefill} \hspace{5mm}
\makebox[0.2\textwidth]{Datum:\enspace\hrulefill}
\vspace{4mm}

\begin{questions}

\question Welche der folgenden Einheiten ist die SI-Einheit für die Wärmeenergie?

\begin{choices}
	\choice Grad Celsius
	\choice Kelvin
	\choice Newton
	\choice Watt
	\choice Joule (correct)
\end{choices}

\vspace{3mm}\question Wenn eine warme und eine kalte Flüssigkeitsmenge gleicher Art gemischt werden, dann hat das Gemisch

\begin{choices}
	\choice eine spezifische Wärmekapazität, die ähnlich groß wie die spezifischen Wärmekapazität der kalten oder der warmen Flüssigkeit ist. (correct)
	\choice die halbe spezifische Wärmekapazität der kalten Flüssigkeit.
	\choice die halbe spezifische Wärmekapazität der warmen Flüssigkeit.
	\choice eine spezifische Wärmekapazität etwa gleich der Summe der spezifischen Wärmekapazitäten der beiden Flüssigkeiten.
	\choice die doppelte spezifische Wärmekapazität der kalten Flüssigkeit.
\end{choices}

\vspace{3mm}\question Welche der folgenden Aussagen ist FALSCH?

\begin{choices}
	\choice Eine Temperatur von 0 K entspricht einer Temperatur von etwa -273,15\(^\circ\)C.
	\choice Für Temperaturdifferenzen erhält man mit der Kelvin- und der Celsiusskala die gleichen Werte.
	\choice Eine Temperaturdifferenz von 1 K entspricht einer Temperaturdifferenz von 1\(^\circ\)C.
	\choice Physikalische Ausdrücke, die die absolute Temperatur enthalten, erfordern Angaben in der Kelvinskala.
	\choice Für die spezifische Wärmekapazität eines Körpers muß man unterschiedliche Werte verwenden, je nachdem, ob man mit Kelvin oder mit Grad Celsius rechnet. (correct)
\end{choices}

\vspace{3mm}\question Durch eine Wärmezufuhr von 4 J werden 2 g einer Flüssigkeit verdampft.Das heißt,

\begin{choices}
	\choice Die Wärmekapazität der Flüssigkeit beträgt 2 J g\(^{-1}\) K\(^{-1}\).
	\choice Die spezifische Verdampfungswärme beträgt 1 J g\(^{-1}\).
	\choice Die spezifische Verdampfungswärme beträgt 2 J g\(^{-1}\). (correct)
	\choice Die spezifische Verdampfungswärme beträgt 0,5 J g\(^{-1}\).
	\choice Die Wärmekapazität der Flüssigkeit beträgt 1 J g\(^{-1}\) K\(^{-1}\).
\end{choices}

\vspace{3mm}\question Welcher Zusammenhang besteht zwischen Wärme, Wärmekapazität und Temperatur? (\(m\): Masse)

\begin{choices}
	\choice \(\Delta Q = m \, (\Delta T)^2\)
	\choice \(\Delta Q = m \Delta T\)
	\choice \(Q = m c\)
	\choice \(\Delta Q = m c \Delta T\) (correct)
	\choice \(Q = m c^2\)
\end{choices}

\vspace{3mm}\end{questions}

\end{document}
