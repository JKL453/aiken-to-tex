\documentclass[11pt]{exam}

\usepackage{geometry}
\geometry{
a4paper,
total={185mm,257mm},
left=10mm,
top=25mm,
bottom=10mm
}

\begin{document}
\setlength{\voffset}{-0.5in}
\setlength{\headsep}{5pt}

\fbox{\fbox{\parbox{8cm}{\centering
\vspace{2mm}
Testat - Versuch C - Stationaerer Strom - 2
\vspace{2mm}
}}}
\hspace{2mm}
\makebox[0.25\textwidth]{Name:\enspace\hrulefill} \hspace{5mm}
\makebox[0.2\textwidth]{Datum:\enspace\hrulefill}
\vspace{4mm}

\begin{questions}

\question Wie groß ist der Gesamtwiderstand einer Parallelschaltung von drei Widerständen mit \(R_1=\mathrm{\frac{1}{3}\,\Omega}\), \(R_2=\mathrm{\frac{1}{4}\,\Omega}\) und \(R_3=\mathrm{\frac{1}{5}\,\Omega}\)?

\begin{choices}
	\choice \(\mathrm{\frac{1}{24}\,\Omega}\)
	\choice \(\mathrm{\frac{1}{12}\,\Omega}\)
	\choice \(\mathrm{\frac{47}{60}\,\Omega}\)
	\choice \(\mathrm{\frac{60}{47}\,\Omega}\)
	\choice \(\mathrm{12\,\Omega}\)
\end{choices}

\vspace{3mm}\question Ein unbekannter Widerstand \(\mathrm{R}\) wird mit einem \(\mathrm{4\,\Omega}\)-Widerstand parallel geschaltet. Der Gesamtwiderstand der Parallelschaltung beträgt dann \(\mathrm{\frac{4}{3}\,\Omega}\). Wie groß ist \(\mathrm{R}\)?

\begin{choices}
	\choice \(\mathrm{5\,\Omega}\)
	\choice \(\mathrm{1\,\Omega}\)
	\choice \(\mathrm{3\,\Omega}\)
	\choice \(\mathrm{2\,\Omega}\)
	\choice \(\mathrm{4\,\Omega}\)
\end{choices}

\vspace{3mm}\question An einen Widerstand \(\mathrm{R=100\,\Omega}\) wird eine Spannung \(\mathrm{U=1000\,mV}\) angelegt. Welche Stromstärke fließt dann durch den Widerstand?

\begin{choices}
	\choice \(\mathrm{100\,mA}\)
	\choice \(\mathrm{10\,A}\)
	\choice \(\mathrm{100\,\mu A}\)
	\choice \(\mathrm{10\,mA}\)
	\choice \(\mathrm{100\,A}\)
\end{choices}

\vspace{3mm}\question Wie groß ist der Gesamtwiderstand einer Reihenschaltung von drei Widerständen mit \(R_1=\mathrm{\frac{1}{3}\,\Omega}\), \(R_2=\mathrm{\frac{1}{4}\,\Omega}\) und \(R_3=\mathrm{\frac{1}{5}\,\Omega}\)?

\begin{choices}
	\choice \(\mathrm{\frac{1}{12}\,\Omega}\)
	\choice \(\mathrm{\frac{47}{60}\,\Omega}\)
	\choice \(\mathrm{\frac{1}{24}\,\Omega}\)
	\choice \(\mathrm{\frac{60}{47}\,\Omega}\)
	\choice \(\mathrm{12\,\Omega}\)
\end{choices}

\vspace{3mm}\question Welche der folgenden Aussagen ist falsch?

\begin{choices}
	\choice Der Innenwiderstand eines Voltmeters sollte möglichst hoch sein.
	\choice Wenn zwei unterschiedliche Widerstände in Reihe geschaltet werden, dann fließt durch beide die gleiche Stromstärke.
	\choice Wenn zwei unterschiedliche Widerstände parallel geschaltet werden, liegt an beiden die gleiche Spannung an.
	\choice Wenn die Spannung an einem ohmschen Widerstand verdoppelt wird, bleibt der Widerstand unverändert.
	\choice Von zwei unterschiedlich langen, aber gleich dicken Drähten aus demselben Material hat der kürzere Draht immer den größeren Widerstand.
\end{choices}

\vspace{3mm}\end{questions}

\end{document}
