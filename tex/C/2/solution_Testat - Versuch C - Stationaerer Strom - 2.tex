\documentclass[11pt]{exam}

\usepackage{amsmath}
\usepackage{graphicx}
\usepackage{geometry}
\usepackage{etoolbox}
\BeforeBeginEnvironment{choices}{\par\nopagebreak\minipage{\linewidth}}
\AfterEndEnvironment{choices}{\endminipage}
\geometry{
a4paper,
total={185mm,257mm},
left=10mm,
top=25mm,
bottom=10mm
}

\begin{document}
\setlength{\voffset}{-0.5in}
\setlength{\headsep}{5pt}

\fbox{\fbox{\parbox{8cm}{\centering
\vspace{2mm}
Testat - Versuch C - Stationaerer Strom - 2
\vspace{2mm}
}}}
\hspace{2mm}
\makebox[0.25\textwidth]{Name:\enspace\hrulefill} \hspace{5mm}
\makebox[0.2\textwidth]{Datum:\enspace\hrulefill}
\vspace{4mm}

\begin{questions}

\question Wie groß ist der Gesamtwiderstand einer Parallelschaltung von drei Widerständen mit \(R_1=\mathrm{\frac{1}{2}\,\Omega}\), \(R_2=\mathrm{\frac{1}{4}\,\Omega}\) und \(R_3=\mathrm{\frac{1}{8}\,\Omega}\)?

\begin{choices}
	\choice \(\mathrm{\frac{1}{64}\,\Omega}\)
	\choice \(\mathrm{\frac{8}{7}\,\Omega}\)
	\choice \(\mathrm{\frac{1}{14}\,\Omega}\) (correct)
	\choice \(\mathrm{\frac{7}{8}\,\Omega}\)
	\choice \(\mathrm{14\,\Omega}\)
\end{choices}

\vspace{3mm}\question Ein unbekannter Widerstand \(\mathrm{R}\) wird mit einem \(\mathrm{4\,\Omega}\)-Widerstand in Reihe an eine Spannungsquelle mit \(\mathrm{U=12\,V}\) angeschlossen. An dem unbekannten Widerstand \(\mathrm{R}\) wird dabei eine Spannung von \(\mathrm{4\,V}\) gemessen. Wie groß ist \(\mathrm{R}\)?

\begin{choices}
	\choice \(\mathrm{5\,\Omega}\)
	\choice \(\mathrm{2\,\Omega}\) (correct)
	\choice \(\mathrm{1\,\Omega}\)
	\choice \(\mathrm{4\,\Omega}\)
	\choice \(\mathrm{3\,\Omega}\)
\end{choices}

\vspace{3mm}\question An einen Widerstand \(\mathrm{R=1000\,\Omega}\) wird eine Spannung \(\mathrm{U=100\,mV}\) angelegt. Welche Stromstärke fließt dann durch den Widerstand?

\begin{choices}
	\choice \(\mathrm{100\,\mu A}\) (correct)
	\choice \(\mathrm{1\,kA}\)
	\choice \(\mathrm{10\,mA}\)
	\choice \(\mathrm{100\,A}\)
	\choice \(\mathrm{10\,A}\)
\end{choices}

\vspace{3mm}\question Wie groß ist der Gesamtwiderstand einer Reihenschaltung von drei Widerständen mit \(R_1=\mathrm{\frac{1}{2}\,\Omega}\), \(R_2=\mathrm{\frac{1}{4}\,\Omega}\) und \(R_3=\mathrm{\frac{1}{8}\,\Omega}\)?

\begin{choices}
	\choice \(\mathrm{\frac{8}{7}\,\Omega}\)
	\choice \(\mathrm{\frac{1}{14}\,\Omega}\)
	\choice \(\mathrm{14\,\Omega}\)
	\choice \(\mathrm{\frac{7}{8}\,\Omega}\) (correct)
	\choice \(\mathrm{\frac{1}{64}\,\Omega}\)
\end{choices}

\vspace{3mm}\question Welche der folgenden Aussagen ist falsch?

\begin{choices}
	\choice Wenn die Spannung an einem ohmschen Widerstand verdoppelt wird, bleibt der Widerstand unverändert.
	\choice Der Gesamtwiderstand einer Reihenschaltung von zwei unterschiedlichen Widerständen ist immer größer als der größere von den beiden Widerständen.
	\choice Der Gesamtwiderstand einer Parallelschaltung von zwei unterschiedlichen Widerständen ist immer größer als der größere von den beiden Widerständen. (correct)
	\choice Von zwei gleich langen, aber unterschiedlich dicken Drähten aus demselben Material hat der dickere Draht immer den geringeren Widerstand.
	\choice Der Gesamtwiderstand einer Reihenschaltung von zwei identischen Widerständen ist vier mal so groß wie der Gesamtwiderstand einer Parallelschaltung aus denselben Widerständen.
\end{choices}

\vspace{3mm}\end{questions}

\end{document}
