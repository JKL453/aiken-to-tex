\documentclass[11pt]{exam}

\usepackage{geometry}
\geometry{
a4paper,
total={185mm,257mm},
left=10mm,
top=25mm,
bottom=10mm
}

\begin{document}
\setlength{\voffset}{-0.5in}
\setlength{\headsep}{5pt}

\fbox{\fbox{\parbox{8cm}{\centering
\vspace{2mm}
Testat - Versuch C - Stationaerer Strom - 3
\vspace{2mm}
}}}
\hspace{2mm}
\makebox[0.25\textwidth]{Name:\enspace\hrulefill} \hspace{5mm}
\makebox[0.2\textwidth]{Datum:\enspace\hrulefill}
\vspace{4mm}

\begin{questions}

\question Wie groß ist der Gesamtwiderstand einer Parallelschaltung von drei Widerständen mit \(R_1=\mathrm{\frac{1}{2}\,\Omega}\), \(R_2=\mathrm{\frac{1}{4}\,\Omega}\) und \(R_3=\mathrm{\frac{1}{8}\,\Omega}\)?

\begin{choices}
	\choice \(\mathrm{\frac{8}{7}\,\Omega}\)
	\choice \(\mathrm{\frac{1}{64}\,\Omega}\)
	\choice \(\mathrm{\frac{7}{8}\,\Omega}\)
	\choice \(\mathrm{\frac{1}{14}\,\Omega}\)
	\choice \(\mathrm{14\,\Omega}\)
\end{choices}

\vspace{3mm}\question Ein unbekannter Widerstand \(\mathrm{R}\) wird mit einem \(\mathrm{4\,\Omega}\)-Widerstand parallel geschaltet. Der Gesamtwiderstand der Parallelschaltung beträgt dann \(\mathrm{\frac{4}{3}\,\Omega}\). Wie groß ist \(\mathrm{R}\)?

\begin{choices}
	\choice \(\mathrm{3\,\Omega}\)
	\choice \(\mathrm{4\,\Omega}\)
	\choice \(\mathrm{1\,\Omega}\)
	\choice \(\mathrm{5\,\Omega}\)
	\choice \(\mathrm{2\,\Omega}\)
\end{choices}

\vspace{3mm}\question An einem Widerstand liegt eine Spannung von \(\mathrm{100\,kV}\) an. Es fließt ein Strom von \(\mathrm{100\,mA}\). Wie groß ist der Widerstand?

\begin{choices}
	\choice \(\mathrm{1000\,M\Omega}\)
	\choice \(\mathrm{100\,M\Omega}\)
	\choice \(\mathrm{1\,\Omega}\)
	\choice \(\mathrm{100\,\Omega}\)
	\choice \(\mathrm{1\,M\Omega}\)
\end{choices}

\vspace{3mm}\question Wie groß ist der Gesamtwiderstand einer Reihenschaltung von drei Widerständen mit \(R_1=\mathrm{\frac{1}{3}\,\Omega}\), \(R_2=\mathrm{\frac{1}{4}\,\Omega}\) und \(R_3=\mathrm{\frac{1}{5}\,\Omega}\)?

\begin{choices}
	\choice \(\mathrm{\frac{60}{47}\,\Omega}\)
	\choice \(\mathrm{\frac{1}{24}\,\Omega}\)
	\choice \(\mathrm{\frac{1}{12}\,\Omega}\)
	\choice \(\mathrm{\frac{47}{60}\,\Omega}\)
	\choice \(\mathrm{12\,\Omega}\)
\end{choices}

\vspace{3mm}\question Welche der folgenden Aussagen ist falsch?

\begin{choices}
	\choice Der Innenwiderstand eines Voltmeters sollte möglichst gering sein.
	\choice Der Gesamtwiderstand einer Parallelschaltung von zwei unterschiedlichen Widerständen ist immer kleiner als der kleinere von den beiden Widerständen.
	\choice Von zwei gleich langen, aber unterschiedlich dicken Drähten aus demselben Material hat der dickere Draht immer den geringeren Widerstand.
	\choice Der Innenwiderstand eines Amperemeters sollte möglichst gering sein.
	\choice Wenn die Spannung an einem ohmschen Widerstand verdoppelt wird, bleibt der Widerstand unverändert.
\end{choices}

\vspace{3mm}\end{questions}

\end{document}
