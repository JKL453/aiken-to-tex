\documentclass[11pt]{exam}

\usepackage{amsmath}
\usepackage{graphicx}
\usepackage{geometry}
\usepackage{etoolbox}
\BeforeBeginEnvironment{choices}{\par\nopagebreak\minipage{\linewidth}}
\AfterEndEnvironment{choices}{\endminipage}
\geometry{
a4paper,
total={185mm,257mm},
left=10mm,
top=25mm,
bottom=10mm
}

\begin{document}
\setlength{\voffset}{-0.5in}
\setlength{\headsep}{5pt}

\fbox{\fbox{\parbox{8cm}{\centering
\vspace{2mm}
Testat - Versuch C - Stationaerer Strom 
\vspace{2mm}
}}}
\hspace{2mm}
\makebox[0.25\textwidth]{Name:\enspace\hrulefill} \hspace{5mm}
\makebox[0.2\textwidth]{Datum:\enspace\hrulefill}
\vspace{4mm}

\begin{questions}

\question Wie groß ist der Gesamtwiderstand einer Parallelschaltung von drei Widerständen mit \(R_1=\mathrm{\frac{1}{2}\,\Omega}\), \(R_2=\mathrm{\frac{1}{4}\,\Omega}\) und \(R_3=\mathrm{\frac{1}{8}\,\Omega}\)?

\begin{choices}
	\choice \(\mathrm{\frac{1}{14}\,\Omega}\) (correct)
	\choice \(\mathrm{\frac{7}{8}\,\Omega}\)
	\choice \(\mathrm{\frac{8}{7}\,\Omega}\)
	\choice \(\mathrm{\frac{1}{64}\,\Omega}\)
	\choice \(\mathrm{14\,\Omega}\)
\end{choices}

\vspace{3mm}\question Ein unbekannter Widerstand \(\mathrm{R}\) wird mit einem \(\mathrm{3\,\Omega}\)-Widerstand parallel geschaltet. Der Gesamtwiderstand der Parallelschaltung beträgt dann \(\mathrm{\frac{3}{4}\,\Omega}\). Wie groß ist \(\mathrm{R}\)?

\begin{choices}
	\choice \(\mathrm{3\,\Omega}\)
	\choice \(\mathrm{1\,\Omega}\) (correct)
	\choice \(\mathrm{4\,\Omega}\)
	\choice \(\mathrm{5\,\Omega}\)
	\choice \(\mathrm{2\,\Omega}\)
\end{choices}

\vspace{3mm}\question An einem Widerstand liegt eine Spannung von \(\mathrm{100\,kV}\) an. Es fließt ein Strom von \(\mathrm{100\,mA}\). Wie groß ist der Widerstand?

\begin{choices}
	\choice \(\mathrm{1\,\Omega}\)
	\choice \(\mathrm{1\,M\Omega}\) (correct)
	\choice \(\mathrm{1000\,M\Omega}\)
	\choice \(\mathrm{100\,M\Omega}\)
	\choice \(\mathrm{100\,\Omega}\)
\end{choices}

\vspace{3mm}\question Wie groß ist der Gesamtwiderstand einer Reihenschaltung von drei Widerständen mit \(R_1=\mathrm{\frac{1}{2}\,\Omega}\), \(R_2=\mathrm{\frac{1}{4}\,\Omega}\) und \(R_3=\mathrm{\frac{1}{8}\,\Omega}\)?

\begin{choices}
	\choice \(\mathrm{\frac{7}{8}\,\Omega}\) (correct)
	\choice \(\mathrm{14\,\Omega}\)
	\choice \(\mathrm{\frac{1}{64}\,\Omega}\)
	\choice \(\mathrm{\frac{8}{7}\,\Omega}\)
	\choice \(\mathrm{\frac{1}{14}\,\Omega}\)
\end{choices}

\vspace{3mm}\question Welche der folgenden Aussagen ist falsch?

\begin{choices}
	\choice Wenn zwei unterschiedliche Widerstände parallel geschaltet werden, dann fließt durch den größeren Widerstand der kleinere Strom.
	\choice Der Gesamtwiderstand einer Reihenschaltung von zwei identischen Widerständen ist doppelt so groß wie der Gesamtwiderstand einer Parallelschaltung aus denselben Widerständen. (correct)
	\choice Wenn zwei unterschiedliche Widerstände in Reihe geschaltet werden, dann fließt durch beide die gleiche Stromstärke.
	\choice Wenn zwei unterschiedliche Widerstände parallel geschaltet werden, liegt an beiden die gleiche Spannung an.
	\choice Wenn zwei unterschiedliche Widerstände in Reihe geschaltet werden, dann liegt am größeren Widerstand die größere Spannung an.
\end{choices}

\vspace{3mm}\end{questions}

\end{document}
