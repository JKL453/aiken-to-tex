\documentclass[11pt]{exam}

\usepackage{geometry}
\geometry{
a4paper,
total={185mm,257mm},
left=10mm,
top=25mm,
bottom=10mm
}

\begin{document}
\setlength{\voffset}{-0.5in}
\setlength{\headsep}{5pt}

\fbox{\fbox{\parbox{8cm}{\centering
\vspace{2mm}
Testat - Versuch C - Stationaerer Strom - 1
\vspace{2mm}
}}}
\hspace{2mm}
\makebox[0.25\textwidth]{Name:\enspace\hrulefill} \hspace{5mm}
\makebox[0.2\textwidth]{Datum:\enspace\hrulefill}
\vspace{4mm}

\begin{questions}

\question Wie groß ist der Gesamtwiderstand einer Parallelschaltung von drei Widerständen mit \(R_1=\mathrm{\frac{1}{3}\,\Omega}\), \(R_2=\mathrm{\frac{1}{4}\,\Omega}\) und \(R_3=\mathrm{\frac{1}{5}\,\Omega}\)?

\begin{choices}
	\choice \(\mathrm{\frac{47}{60}\,\Omega}\)
	\choice \(\mathrm{\frac{1}{24}\,\Omega}\)
	\choice \(\mathrm{\frac{1}{12}\,\Omega}\)
	\choice \(\mathrm{12\,\Omega}\)
	\choice \(\mathrm{\frac{60}{47}\,\Omega}\)
\end{choices}

\vspace{3mm}\question Ein unbekannter Widerstand \(\mathrm{R}\) wird mit einem \(\mathrm{3\,\Omega}\)-Widerstand parallel geschaltet. Der Gesamtwiderstand der Parallelschaltung beträgt dann \(\mathrm{\frac{3}{4}\,\Omega}\). Wie groß ist \(\mathrm{R}\)?

\begin{choices}
	\choice \(\mathrm{2\,\Omega}\)
	\choice \(\mathrm{4\,\Omega}\)
	\choice \(\mathrm{1\,\Omega}\)
	\choice \(\mathrm{5\,\Omega}\)
	\choice \(\mathrm{3\,\Omega}\)
\end{choices}

\vspace{3mm}\question An einen Widerstand \(\mathrm{R=100\,\Omega}\) wird eine Spannung \(\mathrm{U=1000\,mV}\) angelegt. Welche Stromstärke fließt dann durch den Widerstand?

\begin{choices}
	\choice \(\mathrm{10\,A}\)
	\choice \(\mathrm{10\,mA}\)
	\choice \(\mathrm{100\,A}\)
	\choice \(\mathrm{100\,mA}\)
	\choice \(\mathrm{100\,\mu A}\)
\end{choices}

\vspace{3mm}\question Wie groß ist der Gesamtwiderstand einer Reihenschaltung von drei Widerständen mit \(R_1=\mathrm{\frac{1}{2}\,\Omega}\), \(R_2=\mathrm{\frac{1}{3}\,\Omega}\) und \(R_3=\mathrm{\frac{1}{4}\,\Omega}\)?

\begin{choices}
	\choice \(\mathrm{\frac{13}{12}\,\Omega}\)
	\choice \(\mathrm{9\,\Omega}\)
	\choice \(\mathrm{\frac{12}{13}\,\Omega}\)
	\choice \(\mathrm{\frac{1}{24}\,\Omega}\)
	\choice \(\mathrm{\frac{1}{9}\,\Omega}\)
\end{choices}

\vspace{3mm}\question Welche der folgenden Aussagen ist falsch?

\begin{choices}
	\choice Wenn die Spannung an einem ohmschen Widerstand verdoppelt wird, verdoppelt sich der Widerstand.
	\choice Der Gesamtwiderstand einer Parallelschaltung von zwei unterschiedlichen Widerständen ist immer kleiner als der kleinere von den beiden Widerständen.
	\choice Wenn zwei unterschiedliche Widerstände in Reihe geschaltet werden, dann fließt durch beide die gleiche Stromstärke.
	\choice Wenn zwei unterschiedliche Widerstände in Reihe geschaltet werden, dann liegt am größeren Widerstand die größere Spannung an.
	\choice Der Innenwiderstand eines Amperemeters sollte möglichst gering sein.
\end{choices}

\vspace{3mm}\end{questions}

\end{document}
