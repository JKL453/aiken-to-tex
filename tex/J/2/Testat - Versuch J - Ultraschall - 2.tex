\documentclass[11pt]{exam}

\usepackage{geometry}
\geometry{
a4paper,
total={185mm,257mm},
left=10mm,
top=25mm,
bottom=10mm
}

\begin{document}
\setlength{\voffset}{-0.5in}
\setlength{\headsep}{5pt}

\fbox{\fbox{\parbox{8cm}{\centering
\vspace{2mm}
Testat - Versuch J - Ultraschall - 2
\vspace{2mm}
}}}
\hspace{2mm}
\makebox[0.25\textwidth]{Name:\enspace\hrulefill} \hspace{5mm}
\makebox[0.2\textwidth]{Datum:\enspace\hrulefill}
\vspace{4mm}

\begin{questions}

\question Die Schallgeschwindigkeit...

\begin{choices}
	\choice ist bei zwei Schallwellen mit unterschiedlicher Frequenz gleich, wenn sie sich im gleichen Medium ausbreiten.
	\choice ist vom Übertragenden Medium unabhängig
	\choice ist im Vakuum größer als in Luft, da die Luftmoleküle die Schallübertragung dämpfen
	\choice Keine Antwort ist richtig.
	\choice hängt von der Amplitude der Schallwelle ab
\end{choices}

\vspace{3mm}\question Mit welcher Gleichung lässt sich der zurückgelegte Weg \( s \) eines Schallsignals beschreiben?( \( t \) Laufzeit des Signals, \( c \) Schallgeschwindigkeit )

\begin{choices}
	\choice \( s=c \cdot t \)
	\choice \( s=c^2 \cdot t \)
	\choice \( s = c \cdot t^2 \)
	\choice \( s= \frac{t}{c} \)
	\choice \( s= \frac{c}{t} \)
\end{choices}

\vspace{3mm}\question Welche Wellenlänge \( \lambda \) hat eine Schallwelle in Luft mit der Frequenz \( f = \mathrm{1~Hz} \) ?

\begin{choices}
	\choice 34,3 m
	\choice 3,43 m
	\choice 1 m
	\choice 100 m
	\choice 343 m
\end{choices}

\vspace{3mm}\question Die vom Menschen empfundene Lautstärke einer Schallwelle

\begin{choices}
	\choice wird in Dezibel (dB) angegeben.
	\choice ist antiproportional zur Schallintensität.
	\choice hängt nicht von der Frequenz der Schallwelle ab.
	\choice Die verbleibenden Alle Aussagen sind falsch.
	\choice ist proportional zur Schallintensität.
\end{choices}

\vspace{3mm}\question Welche Aussage zur Resonanzfrequenz ist richtig?

\begin{choices}
	\choice Wird ein schwingendes System mit seiner Resonanzfrequenz angeregt, erhöht sich die Amplitude der Schwingung sehr stark.
	\choice Die Resonanzfreuqenz ist die Frequenz, die entsteht, wenn zwei Schallwellen sich phasengleich überlagern.
	\choice keine Antwort ist richtig.
	\choice Die Resonanzfrequenz ist die Frequenz, bei der eine Totalrefexion erfolgt.
	\choice Trifft eine Schallwelle auf einen Körper (z.B. eine Wand), so entsteht durch Reflexion ein Echo. Mit Resonanzfrequenz bezeichnet man die Frequenz des Echos.
\end{choices}

\vspace{3mm}\end{questions}

\end{document}
