\documentclass[11pt]{exam}

\usepackage{geometry}
\geometry{
a4paper,
total={185mm,257mm},
left=10mm,
top=25mm,
bottom=10mm
}

\begin{document}
\setlength{\voffset}{-0.5in}
\setlength{\headsep}{5pt}

\fbox{\fbox{\parbox{8cm}{\centering
\vspace{2mm}
Testat - Versuch J - Ultraschall
\vspace{2mm}
}}}
\hspace{2mm}
\makebox[0.25\textwidth]{Name:\enspace\hrulefill} \hspace{5mm}
\makebox[0.2\textwidth]{Datum:\enspace\hrulefill}
\vspace{4mm}

\begin{questions}

\question Die Schallgeschwindigkeit...

\begin{choices}
	\choice ist vom Übertragenden Medium unabhängig
	\choice ist bei zwei Schallwellen mit unterschiedlicher Frequenz gleich, wenn sie sich im gleichen Medium ausbreiten.
	\choice ist im Vakuum größer als in Luft, da die Luftmoleküle die Schallübertragung dämpfen
	\choice Keine Antwort ist richtig.
	\choice hängt von der Amplitude der Schallwelle ab
\end{choices}

\vspace{3mm}\question Mit welcher Gleichung lässt sich der zurückgelegte Weg \( s \) eines Schallsignals beschreiben?( \( t \) Laufzeit des Signals, \( c \) Schallgeschwindigkeit )

\begin{choices}
	\choice \( s=c \cdot t \)
	\choice \( s=c^2 \cdot t \)
	\choice \( s= \frac{c}{t} \)
	\choice \( s = c \cdot t^2 \)
	\choice \( s= \frac{t}{c} \)
\end{choices}

\vspace{3mm}\question Eine Schallwelle in Luft mit der Periodendauer \( T= \mathrm{1~s} \) hat eine Wellenlänge \( \lambda \) von ...?

\begin{choices}
	\choice 34,3 km/h
	\choice 34,3 m
	\choice 343 km/h
	\choice 343 m
	\choice 3,43 m
\end{choices}

\vspace{3mm}\question Bei welcher Frequenz \( f \) befindet sich bei einem gesunden menschlichen Durchschnittsohr die obere Hörschwelle?

\begin{choices}
	\choice 20 kHz
	\choice 16 MHz
	\choice 20 Hz
	\choice 16 Hz
	\choice 20 MHz
\end{choices}

\vspace{3mm}\question Aussage zur Schallausbreitung ist falsch?

\begin{choices}
	\choice Bleibt das Medium unverändert, ist die Wellenlänge umso größer, je kleiner die Frequenz ist
	\choice Bleibt das Medium unverändert, ist die Wellenlänge umso kleiner, je größer die Frequenz ist
	\choice Ändert sich die Ausbreitungsgeschwindigkeit an der Grenze zu einem anderen Medium, so bleibt die Wellenlänge einer Schallwelle konstant
	\choice Bleibt das Medium unverändert, ist die Schallgeschwindigkeit konstant
	\choice Ändert sich die Ausbreitungsgeschwindigkeit an der Grenze zu einem anderen Medium, so bleibt die Frequenz einer Schallwelle konstant
\end{choices}

\vspace{3mm}\end{questions}

\end{document}
