\documentclass[11pt]{exam}

\usepackage{geometry}
\geometry{
a4paper,
total={185mm,257mm},
left=10mm,
top=25mm,
bottom=10mm
}

\begin{document}
\setlength{\voffset}{-0.5in}
\setlength{\headsep}{5pt}

\fbox{\fbox{\parbox{8cm}{\centering
\vspace{2mm}
Testat - Versuch J - Ultraschall
\vspace{2mm}
}}}
\hspace{2mm}
\makebox[0.25\textwidth]{Name:\enspace\hrulefill} \hspace{5mm}
\makebox[0.2\textwidth]{Datum:\enspace\hrulefill}
\vspace{4mm}

\begin{questions}

\question Welche Größe ändert sich bei einer Schallwelle periodisch?

\begin{choices}
	\choice Schallgeschwindigkeit
	\choice Frequenz
	\choice Periodendauer
	\choice Druck
	\choice Amplitude
\end{choices}

\vspace{3mm}\question Welche der folgenden Formeln für die Schallgeschwindigkeit ist richtig?( \( c \) Schallgeschwindigkeit, \(T \) Periodendauer, \( f \) Frequenz, \( \lambda \) Wellenlänge)

\begin{choices}
	\choice \( c=T \cdot \lambda \)
	\choice \( c=\frac{T}{\lambda} \)
	\choice \( c=T \cdot f \)
	\choice \( c=f \cdot \lambda \)
	\choice \( c=\frac{f}{\lambda} \)
\end{choices}

\vspace{3mm}\question Welche Wellenlänge \( \lambda \) hat eine Schallwelle in Luft mit der Frequenz \( f = \mathrm{1~Hz} \) ?

\begin{choices}
	\choice 3,43 m
	\choice 343 m
	\choice 34,3 m
	\choice 1 m
	\choice 100 m
\end{choices}

\vspace{3mm}\question In welchem Bereich liegen die für den Menschen hörbaren Frequenzen?

\begin{choices}
	\choice 2 kHz bis 200 kHz
	\choice 20 Hz bis 20 kHz
	\choice 20 kHz bis 2 kHz
	\choice 20 kHz bis 20 MHz
	\choice 20 Hz bis 200 Hz
\end{choices}

\vspace{3mm}\question Welche Aussage zur Resonanzfrequenz ist richtig?

\begin{choices}
	\choice Die Resonanzfrequenz ist die Frequenz, bei der eine Totalrefexion erfolgt.
	\choice Die Resonanzfreuqenz ist die Frequenz, die entsteht, wenn zwei Schallwellen sich phasengleich überlagern.
	\choice keine Antwort ist richtig.
	\choice Wird ein schwingendes System mit seiner Resonanzfrequenz angeregt, erhöht sich die Amplitude der Schwingung sehr stark.
	\choice Trifft eine Schallwelle auf einen Körper (z.B. eine Wand), so entsteht durch Reflexion ein Echo. Mit Resonanzfrequenz bezeichnet man die Frequenz des Echos.
\end{choices}

\vspace{3mm}\end{questions}

\end{document}
