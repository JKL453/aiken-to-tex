\documentclass[11pt]{exam}

\usepackage{geometry}
\geometry{
a4paper,
total={185mm,257mm},
left=10mm,
top=25mm,
bottom=10mm
}

\begin{document}
\setlength{\voffset}{-0.5in}
\setlength{\headsep}{5pt}

\fbox{\fbox{\parbox{8cm}{\centering
\vspace{2mm}
Testat - Versuch J - Ultraschall - 3
\vspace{2mm}
}}}
\hspace{2mm}
\makebox[0.25\textwidth]{Name:\enspace\hrulefill} \hspace{5mm}
\makebox[0.2\textwidth]{Datum:\enspace\hrulefill}
\vspace{4mm}

\begin{questions}

\question Bei welcher Frequenz \( f \) befindet sich bei einem gesunden menschlichen Durchschnittsohr die untere Hörschwelle?

\begin{choices}
	\choice 20 Hz
	\choice 16 kHz
	\choice 20 kHz
	\choice 16 MHz
	\choice 2000 Hz
\end{choices}

\vspace{3mm}\question Welche Zuordnung von physikalischer Größe und Einheit ist falsch?

\begin{choices}
	\choice Lautstärkepegel - Phon
	\choice Wellenlänge - Meter (m)
	\choice Schallgeschwindigkeit - Watt pro Quadratmeter (\( \mathrm{W m^{-2}} \) )
	\choice Schalldruck - Pascal (Pa)
	\choice Frequenz - Hertz (Hz)
\end{choices}

\vspace{3mm}\question Eine Schallwelle in Luft mit der Periodendauer \( T= \mathrm{1~s} \) hat eine Wellenlänge \( \lambda \) von ...?

\begin{choices}
	\choice 34,3 km/h
	\choice 3,43 m
	\choice 343 m
	\choice 34,3 m
	\choice 343 km/h
\end{choices}

\vspace{3mm}\question Bei welcher Frequenz \( f \) befindet sich bei einem gesunden menschlichen Durchschnittsohr die obere Hörschwelle?

\begin{choices}
	\choice 20 kHz
	\choice 16 MHz
	\choice 20 Hz
	\choice 16 Hz
	\choice 20 MHz
\end{choices}

\vspace{3mm}\question Welche Aussagen zum Piezoeffekt sind zutreffend?	* Durch Anlegen einer elektrischen Spannung verformt sich ein Piezo-Kristall.	* Wird ein Piezo-Kristall mechanisch verformt, entsteht ein Spannungssignal.	* Die Form eines Piezo-Kristalls ist unabhängig von der anliegenden Spannung.

\begin{choices}
	\choice Nur Aussage 1 ist falsch.
	\choice Nur Aussage 3 ist richtig.
	\choice Nur Aussage 1 ist richtig.
	\choice Nur Aussage 3 ist falsch.
	\choice Nur Aussage 2 ist richtig.
\end{choices}

\vspace{3mm}\end{questions}

\end{document}
