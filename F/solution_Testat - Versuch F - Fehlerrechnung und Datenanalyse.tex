\documentclass[11pt]{exam}

\usepackage{amsmath}
\usepackage{graphicx}
\usepackage{geometry}
\usepackage{etoolbox}
\BeforeBeginEnvironment{choices}{\par\nopagebreak\minipage{\linewidth}}
\AfterEndEnvironment{choices}{\endminipage}
\geometry{
a4paper,
total={185mm,257mm},
left=10mm,
top=25mm,
bottom=10mm
}

\begin{document}
\setlength{\voffset}{-0.5in}
\setlength{\headsep}{5pt}

\fbox{\fbox{\parbox{8cm}{\centering
\vspace{2mm}
Testat - Versuch F - Fehlerrechnung
\vspace{2mm}
}}}
\hspace{2mm}
\makebox[0.25\textwidth]{Name:\enspace\hrulefill} \hspace{5mm}
\makebox[0.2\textwidth]{Datum:\enspace\hrulefill}
\vspace{4mm}

\begin{questions}


% Question 1
\question Welche Aussage ist NICHT korrekt? \\

\begin{choices}
	\choice Der mittlere Fehler des Mittelwertes lässt sich durch die Erhöhung des Stichprobenumfangs verringern.
	\choice Die Streuung einer Messreihe wird durch die Standardabweichung beschrieben.
	\choice Die Standardabweichung ändert sich durch die Veränderung des Stichprobenumfangs systematisch. (correct)
	\choice Bei Vorliegen einer Poisson-Statistik ist die Standardabweichung \( \sigma = \sqrt{\mu} \).
	\choice Der Fehler einer Bestgeraden kann grafisch mit Hilfe von Grenzgeraden abgeschätzt werden.
\end{choices}

% Question 2
\vspace{12mm}\question Der mittlere Fehler des Mittelwertes \dots

a) \dots hängt nicht von der Anzahl der Messungen ab. \\
b) \dots wird mithilfe der Standardabweichung berechnet.\\
c) \dots hängt nicht von der Qualität des Messverfahrens ab.\\
d) \dots ist ein Maß für die Qualität der Messung.\\


\begin{choices}
	\choice Alle Aussagen sind richtig.
	\choice Aussagen a, b und c sind richtig.
	\choice Aussagen b und d sind richtig. (correct)
	\choice Nur Aussage d ist richtig.
	\choice Aussagen a und b sind richtig.
\end{choices}

% Question 3
\vspace{12mm}\question Die Standardabweichung \dots 

a) \dots sagt aus, dass mit einer Wahrscheinlichkeit von 68\% ein weiterer Messwert innerhalb der Fehlerbreite vom wahren Wert der Messgröße entfernt liegen wird. \\
b) \dots hängt von der Anzahl der Messungen ab. \\
c) \dots ist ein Maß für die Qualität eines Messverfahrens. \\
d) \dots kann durch Erhöhung der Anzahl an Einzelmessungen nicht systematisch verringert werden.\\

\begin{choices}
	\choice Aussagen a, c und d sind richtig. (correct)
	\choice Aussagen a, b und c sind richtig.
	\choice Nur Aussage c ist richtig.
	\choice Nur Aussage a ist richtig.
	\choice Alle Aussagen sind richtig.
\end{choices}



\newpage

% Question 4
\vspace{12mm}\question Welche ist KEINE Eigenschaft und KEIN Beispiel systematischer Messfehler? \\
Systematische Messfehler \dots \\

\begin{choices}
	\choice \dots können durch die Messapparatur verursacht werden.
	\choice \dots sind reproduzierbar.
	\choice \dots können durch eine falsche Kalibrierung des Messgerätes auftreten.
	\choice \dots streuen in Betrag und Vorzeichen. (correct)
\end{choices}


% Question 5
\vspace{12mm}\question Welche ist KEINE Eigenschaft und KEIN Beispiel zufälliger Messfehler? \\
Zufällige Messfehler \dots \\

\begin{choices}
	\choice \dots unterliegen statistischen Schwankungen.
	\choice \dots können durch sich ändernde Ableseeigenheiten/-winkel des Versuchsdurchführenden auftreten.
	\choice \dots können durch Digitalisierungsrauschen auftreten.
	\choice \dots treten spontan auf.
	\choice \dots können nicht durch eine Vergrößerung des Stichprobenumfangs reduziert werden. (correct)
\end{choices}

\vspace{3mm}\end{questions}

\end{document}
